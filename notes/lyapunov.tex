\documentclass[reqno]{amsart}
%\usepackage{hyperref}
\usepackage{fullpage}
\usepackage{amsrefs}
\usepackage{verbatim}
\usepackage{tikz}
\usepackage{graphicx}

\newif\ifscreen
\newif\iftwo
\newif\ifshowall
\newif\ifshowkeys
\screenfalse
\twotrue
\showallfalse
%\showkeystrue
\showkeysfalse

\input xypic

\ifshowkeys
\newcommand{\lbl}[1]{\label{#1}\textup{[\texttt{#1}]}\ \\}
\else
\newcommand{\lbl}{\label}
\fi

\definecolor{olivegreen}{cmyk}{0.64,0,0.95,0.40}
\definecolor{rawsienna}{cmyk}{0,0.72,1,0.45}
\definecolor{lightgreen}{rgb}{0.85,1.0,0.85}

\newcommand{\GREENYELLOW}[1]{{\color{greenyellow}#1}}
\newcommand{\YELLOW}[1]{{\color{yellow}#1}}
\newcommand{\YLW}[1]{{\color{yellow}#1}}
\newcommand{\GOLDENROD}[1]{{\color{goldenrod}#1}}
\newcommand{\DANDELION}[1]{{\color{dandelion}#1}}
\newcommand{\APRICOT}[1]{{\color{apricot}#1}}
\newcommand{\PEACH}[1]{{\color{peach}#1}}
\newcommand{\MELON}[1]{{\color{melon}#1}}
\newcommand{\YELLOWORANGE}[1]{{\color{yelloworange}#1}}
\newcommand{\ORANGE}[1]{{\color{orange}#1}}
\newcommand{\BURNTORANGE}[1]{{\color{burntorange}#1}}
\newcommand{\BITTERSWEET}[1]{{\color{bittersweet}#1}}
\newcommand{\REDORANGE}[1]{{\color{redorange}#1}}
\newcommand{\MAHOGANY}[1]{{\color{mahogany}#1}}
\newcommand{\MAROON}[1]{{\color{maroon}#1}}
\newcommand{\BRICKRED}[1]{{\color{brickred}#1}}
\newcommand{\RED}[1]{{\color{red}#1}}
\newcommand{\ORANGERED}[1]{{\color{orangered}#1}}
\newcommand{\RUBINERED}[1]{{\color{rubinered}#1}}
\newcommand{\WILDSTRAWBERRY}[1]{{\color{wildstrawberry}#1}}
\newcommand{\SALMON}[1]{{\color{salmon}#1}}
\newcommand{\CARNATIONPINK}[1]{{\color{carnationpink}#1}}
\newcommand{\MAGENTA}[1]{{\color{magenta}#1}}
\newcommand{\VIOLETRED}[1]{{\color{violetred}#1}}
\newcommand{\RHODAMINE}[1]{{\color{rhodamine}#1}}
\newcommand{\MULBERRY}[1]{{\color{mulberry}#1}}
\newcommand{\REDVIOLET}[1]{{\color{redviolet}#1}}
\newcommand{\FUCHSIA}[1]{{\color{fuchsia}#1}}
\newcommand{\LAVENDER}[1]{{\color{lavender}#1}}
\newcommand{\THISTLE}[1]{{\color{thistle}#1}}
\newcommand{\ORCHID}[1]{{\color{orchid}#1}}
\newcommand{\DARKORCHID}[1]{{\color{darkorchid}#1}}
\newcommand{\PURPLE}[1]{{\color{purple}#1}}
\newcommand{\PLUM}[1]{{\color{plum}#1}}
\newcommand{\VIOLET}[1]{{\color{violet}#1}}
\newcommand{\ROYALPURPLE}[1]{{\color{royalpurple}#1}}
\newcommand{\BLUEVIOLET}[1]{{\color{blueviolet}#1}}
\newcommand{\PERIWINKLE}[1]{{\color{periwinkle}#1}}
\newcommand{\CADETBLUE}[1]{{\color{cadetblue}#1}}
\newcommand{\CORNFLOWERBLUE}[1]{{\color{cornflowerblue}#1}}
\newcommand{\MIDNIGHTBLUE}[1]{{\color{midnightblue}#1}}
\newcommand{\NAVYBLUE}[1]{{\color{navyblue}#1}}
\newcommand{\ROYALBLUE}[1]{{\color{royalblue}#1}}
\newcommand{\BLU}[1]{{\color{blue}#1}}
\newcommand{\BLUE}[1]{{\color{blue}#1}}
\newcommand{\CERULEAN}[1]{{\color{cerulean}#1}}
\newcommand{\CYAN}[1]{{\color{cyan}#1}}
\newcommand{\PROCESSBLUE}[1]{{\color{processblue}#1}}
\newcommand{\SKYBLUE}[1]{{\color{skyblue}#1}}
\newcommand{\TURQUOISE}[1]{{\color{turquoise}#1}}
\newcommand{\TEALBLUE}[1]{{\color{tealblue}#1}}
\newcommand{\AQUAMARINE}[1]{{\color{aquamarine}#1}}
\newcommand{\BLUEGREEN}[1]{{\color{bluegreen}#1}}
\newcommand{\EMERALD}[1]{{\color{emerald}#1}}
\newcommand{\JUNGLEGREEN}[1]{{\color{junglegreen}#1}}
\newcommand{\SEAGREEN}[1]{{\color{seagreen}#1}}
\newcommand{\GREEN}[1]{{\color{green}#1}}
\newcommand{\FORESTGREEN}[1]{{\color{forestgreen}#1}}
\newcommand{\PINEGREEN}[1]{{\color{pinegreen}#1}}
\newcommand{\LIMEGREEN}[1]{{\color{limegreen}#1}}
\newcommand{\YELLOWGREEN}[1]{{\color{yellowgreen}#1}}
\newcommand{\SPRINGGREEN}[1]{{\color{springgreen}#1}}
\newcommand{\OLIVEGREEN}[1]{{\color{olivegreen}#1}}
\newcommand{\OLG}[1]{{\color{olivegreen}#1}}
\newcommand{\RAWSIENNA}[1]{{\color{rawsienna}#1}}
\newcommand{\SEPIA}[1]{{\color{sepia}#1}}
\newcommand{\BROWN}[1]{{\color{brown}#1}}
\newcommand{\TAN}[1]{{\color{tan}#1}}
\newcommand{\GRAY}[1]{{\color{gray}#1}}
\newcommand{\LGRAY}[1]{{\color{gray!40}#1}}
\newcommand{\WHITE}[1]{{\color{white}#1}}
\newcommand{\BLACK}[1]{{\color{black}#1}}

\newcommand{\bbm}       {\left[\begin{matrix}}
\newcommand{\ebm}       {\end{matrix}\right]}
\newcommand{\bsm}       {\left[\begin{smallmatrix}}
\newcommand{\esm}       {\end{smallmatrix}\right]}
\newcommand{\bpm}       {\begin{pmatrix}}
\newcommand{\epm}       {\end{pmatrix}}
\newcommand{\bcf}[2]{\left(\begin{array}{c}{#1}\\{#2}\end{array}\right)}

\newcommand{\adj}       {\operatorname{adj}}
\newcommand{\ann}       {\operatorname{ann}}
\newcommand{\diag}      {\operatorname{diag}}
\newcommand{\img}       {\operatorname{img}}
\newcommand{\rnk}       {\operatorname{rank}}
\newcommand{\sgn}       {\operatorname{sgn}}
\newcommand{\spn}       {\operatorname{span}}
\newcommand{\trc}       {\operatorname{trace}}

\newcommand{\pp}{\hphantom{+}}
\newcommand{\tm}{\times}
\newcommand{\sse}{\subseteq}
\newcommand{\st}{\;|\;}
\newcommand{\sm}{\setminus}
\newcommand{\iffa}      {\Leftrightarrow}
\newcommand{\xra}{\xrightarrow}
\newcommand{\xla}{\xleftarrow}

\newcommand{\half}{\tfrac{1}{2}}

\newcommand{\N}         {{\mathbb{N}}}
\newcommand{\Z}         {{\mathbb{Z}}}
\newcommand{\Q}         {{\mathbb{Q}}}
\newcommand{\R}         {{\mathbb{R}}}
\newcommand{\C}         {{\mathbb{C}}}

\newcommand{\va}        {\mathbf{a}}
\newcommand{\vb}        {\mathbf{b}}
\newcommand{\vc}        {\mathbf{c}}
\newcommand{\vd}        {\mathbf{d}}
\newcommand{\ve}        {\mathbf{e}}
\newcommand{\vf}        {\mathbf{f}}
\newcommand{\vg}        {\mathbf{g}}
\newcommand{\vh}        {\mathbf{h}}
\newcommand{\vi}        {\mathbf{i}}
\newcommand{\vj}        {\mathbf{j}}
\newcommand{\vk}        {\mathbf{k}}
\newcommand{\vl}        {\mathbf{l}}
\newcommand{\vm}        {\mathbf{m}}
\newcommand{\vn}        {\mathbf{n}}
\newcommand{\vo}        {\mathbf{o}}
\newcommand{\vp}        {\mathbf{p}}
\newcommand{\vq}        {\mathbf{q}}
\newcommand{\vr}        {\mathbf{r}}
\newcommand{\vs}        {\mathbf{s}}
\newcommand{\vt}        {\mathbf{t}}
\newcommand{\vu}        {\mathbf{u}}
\newcommand{\vv}        {\mathbf{v}}
\newcommand{\vw}        {\mathbf{w}}
\newcommand{\vx}        {\mathbf{x}}
\newcommand{\vy}        {\mathbf{y}}
\newcommand{\vz}        {\mathbf{z}}

\newcommand{\vA}        {\mathbf{A}}
\newcommand{\vB}        {\mathbf{B}}
\newcommand{\vC}        {\mathbf{C}}
\newcommand{\vD}        {\mathbf{D}}
\newcommand{\vE}        {\mathbf{E}}
\newcommand{\vF}        {\mathbf{F}}
\newcommand{\vG}        {\mathbf{G}}
\newcommand{\vH}        {\mathbf{H}}
\newcommand{\vI}        {\mathbf{I}}
\newcommand{\vJ}        {\mathbf{J}}
\newcommand{\vK}        {\mathbf{K}}
\newcommand{\vL}        {\mathbf{L}}
\newcommand{\vM}        {\mathbf{M}}
\newcommand{\vN}        {\mathbf{N}}
\newcommand{\vO}        {\mathbf{O}}
\newcommand{\vP}        {\mathbf{P}}
\newcommand{\vQ}        {\mathbf{Q}}
\newcommand{\vR}        {\mathbf{R}}
\newcommand{\vS}        {\mathbf{S}}
\newcommand{\vT}        {\mathbf{T}}
\newcommand{\vU}        {\mathbf{U}}
\newcommand{\vV}        {\mathbf{V}}
\newcommand{\vW}        {\mathbf{W}}
\newcommand{\vX}        {\mathbf{X}}
\newcommand{\vY}        {\mathbf{Y}}
\newcommand{\vZ}        {\mathbf{Z}}

\newcommand{\al}        {\alpha}
\newcommand{\bt}        {\beta} 
\newcommand{\gm}        {\gamma}
\newcommand{\dl}        {\delta}
\newcommand{\ep}        {\epsilon}
\newcommand{\zt}        {\zeta}
\newcommand{\et}        {\eta}
\newcommand{\tht}       {\theta}
\newcommand{\io}        {\iota}
\newcommand{\kp}        {\kappa}
\newcommand{\lm}        {\lambda}
\newcommand{\ph}        {\phi}
\newcommand{\ch}        {\chi}
\newcommand{\ps}        {\psi}
\newcommand{\rh}        {\rho}
\newcommand{\sg}        {\sigma}
\newcommand{\om}        {\omega}

\newcommand{\Gm}        {\Gamma}
\newcommand{\Dl}        {\Delta}

\newcommand{\CA}        {\mathcal{A}}
\newcommand{\CB}        {\mathcal{B}}
\newcommand{\CC}        {\mathcal{C}}
\newcommand{\CD}        {\mathcal{D}}
\newcommand{\CE}        {\mathcal{E}}
\newcommand{\CF}        {\mathcal{F}}
\newcommand{\CG}        {\mathcal{G}}
\newcommand{\CH}        {\mathcal{H}}
\newcommand{\CI}        {\mathcal{I}}
\newcommand{\CJ}        {\mathcal{J}}
\newcommand{\CK}        {\mathcal{K}}
\newcommand{\CL}        {\mathcal{L}}
\newcommand{\CM}        {\mathcal{M}}
\newcommand{\CN}        {\mathcal{N}}
\newcommand{\CO}        {\mathcal{O}}
\newcommand{\CP}        {\mathcal{P}}
\newcommand{\CQ}        {\mathcal{Q}}
\newcommand{\CR}        {\mathcal{R}}
\newcommand{\CS}        {\mathcal{S}}
\newcommand{\CT}        {\mathcal{T}}
\newcommand{\CU}        {\mathcal{U}}
\newcommand{\CV}        {\mathcal{V}}
\newcommand{\CW}        {\mathcal{W}}
\newcommand{\CX}        {\mathcal{X}}
\newcommand{\CY}        {\mathcal{Y}}
\newcommand{\CZ}        {\mathcal{Z}}


\newcommand{\ov}        {\overline}
\newcommand{\ip}[1]     {\langle #1\rangle}
\renewcommand{\ss}      {\scriptstyle}

\renewcommand{\:}       {\colon}

\newcommand{\barmat}[2]{\left[\begin{array}{c|c}\!\!\raisebox{0pt}[0.45cm][0.35cm]{$#1$} & \raisebox{0pt}[0.45cm][0.35cm]{$#2$}\!\!\end{array}\right]}

\newcommand{\eqpair}[4]{\begin{array}{rl} #1 &= #2 \\ #3 &= #4\end{array}}

\newcommand{\han}[1]{\begin{CJK*}{UTF8}{zhsong}\BLUE{#1}\end{CJK*}}
\newcommand{\bhan}[1]{(\begin{CJK*}{UTF8}{zhsong}\BLUE{#1}\end{CJK*})}

\newcommand{\EMPH}[1]{\emph{\RED{#1}}}
\newcommand{\DEFN}[1]{\emph{\PURPLE{#1}}}
\newcommand{\VEC}[1]    {\mathbf{#1}}

\newcommand{\ghost}{{\tiny $\color[rgb]{1,1,1}.$}}

\newcommand{\reminderbar}{\par\medskip\par\hrule\par\medskip\par}

\newcommand{\uc}{\uncover}

\newcommand{\bbox}[1]{
\[ \mbox{\begin{tikzpicture}%
   \draw(0,0) node[draw,thick,olivegreen,rectangle] {\color{black} #1};%
  \end{tikzpicture}} \]
}

\newcommand{\cbox}[1]{
\begin{center}\begin{tikzpicture}%
   \draw(0,0) node[draw,thick,olivegreen,rectangle] {\color{black} #1};%
\end{tikzpicture}\end{center}
}



\newtheorem{theorem}{Theorem}
\newtheorem{conj}[theorem]{Conjecture}
\newtheorem{lemma}[theorem]{Lemma}
\newtheorem{proposition}[theorem]{Proposition}
\newtheorem{corollary}[theorem]{Corollary}
\theoremstyle{definition}
\newtheorem{assumption}[theorem]{Assumption}
\newtheorem{remark}[theorem]{Remark}
\newtheorem{predefinition}[theorem]{Predefinition}
\newtheorem{definition}[theorem]{Definition}
\newtheorem{example}[theorem]{Example}
\newtheorem{algorithm}[theorem]{Algorithm}
\newtheorem{method}[theorem]{Method}
\newtheorem{fact}[theorem]{Fact}
% Exercises are numbered separately
\newtheorem{exercise}{Exercise}[section]

%\renewenvironment{solution}{\SolutionAtEnd}{\endSolutionAtEnd}
%\renewenvironment{solution}{\SolutionHidden}{\endSolutionHidden}

\newwrite\refs
\openout\refs=\jobname.refs
\makeatletter
\renewcommand\@setref[3]{%
        \ifx#1\relax
                \write\refs{'#3' \thepage\space undefined}%
                \protect \G@refundefinedtrue
                \nfss@text{\reset@font\bfseries ??}%
                \@latex@warning{Reference `#3' on page \thepage\space
                                undefined}%
        \else
                \begingroup
                \count@\expandafter\@secondoftwo#1\relax
                \ifnum\c@page<\count@
                        \write\refs{'#3' \thepage\space
                                    \expandafter\@secondoftwo#1}%
                \fi
                \endgroup
                \expandafter#2#1\null
        \fi
}
\makeatother

\begin{document}

\title{Some proofs for Lyapunov theory}
\maketitle

We first recall some background.  Convergence of real numbers is
defined as follows:

\begin{definition}
 A sequence $(x_n)$ of real numbers converges to a real number $a$ if
 for all $\ep>0$, there is a natural number $N$ such that
 $|x_n-a|<\ep$ for all $n\geq N$.
\end{definition}

We need to generalise this to cover sequences in $\R^2$ rather than
$\R$. 

\begin{definition}
 Suppose we have a sequence of points $(x_n,y_n)\in\R^2$, and another
 point $(a,b)\in\R^2$.  We say that the sequence converges to $(a,b)$
 if for all $\ep>0$ there exists $N$ such that
 $\|(x_n,y_n)-(a,b)\|<\ep$ for all $n\geq N$.
\end{definition}

This can easily be related to convergence in $\R$:
\begin{lemma}
 Suppose we have a sequence of points $(x_n,y_n)\in\R^2$, and another
 point $(a,b)\in\R^2$.  Then $(x_n,y_n)$ converges to $(a,b)$ if and
 only if $x_n$ converges to $a$ and $y_n$ converges to $b$.
\end{lemma}
\begin{proof}
 Suppose that $(x_n,y_n)\to (a,b)$.  Given $\ep>0$, there
 exists $N$ such that when $n\geq N$ we have
 $\|(x_n,y_n)-(a,b)\|<\ep$.  However, we also have 
 \[ |x_n-a| \leq \sqrt{|x_n-a|^2+|y_n-b|^2} = \|(x_n,y_n)-(a,b)\|, \]
 so $|x_n-a|<\ep$.  This proves that $x_n\to a$, and similarly
 $y_n\to b$.

 Conversely, suppose that $x_n\to a$ and $y_n\to b$.  Given $\ep>0$ we
 can choose $L$ such that $|x_n-a|<\ep/\sqrt{2}$ for $n\geq L$, and we
 can choose $M$ such that $|y_n-b|<\ep/\sqrt{2}$ for $n\geq M$.  Put
 $N=\max(L,M)$.  For $n\geq N$ we have $|x_n-a|<\ep/\sqrt{2}$ and
 $|y_n-b|<\ep/\sqrt{2}$, so 
 \[ \|(x_n,y_n)-(a,b)\| = \sqrt{|x_n-a|^2+|y_n-b|^2} <
      \sqrt{\ep^2/2+\ep^2/2} = \ep,
 \]
 as required.
\end{proof}

The following result is standard:
\begin{theorem}[Bolzano-Weierstrass]
 Let $(x_n)$ be a sequence of real numbers, and suppose that there is
 a constant $C$ such that $|x_n|\leq C$ for all $n$.  Then there is a
 subsequence $(x_{n_k})$ (with $n_0<n_1<n_2<\dotsb $) and a real
 number $a$ such that $x_{n_k}\to a$.
\end{theorem}

We will need the two-dimensional version:
\begin{corollary}
 Let $(x_n,y_n)$ be a sequence in $\R^2$, and suppose that there is
 a constant $C$ such that $\|(x_n,y_n)\|\leq C$ for all $n$.  Then
 there is a subsequence $(x_{n_k},y_{n_k})$ (with
 $n_0<n_1<n_2<\dotsb $) and a point $(a,b)\in\R^2$ such that
 $(x_{n_k},y_{n_k})\to(a,b)$. 
\end{corollary}
\begin{proof}
 By the original Bolzano-Weierstrass Theorem, we can find a real
 number $a$ and an increasing sequence of indices $p_k$ such that
 $x_{p_k}\to a$.  Now the sequence $y_{p_k}$ is again bounded by $C$,
 so we can find a real number $b$ and an increasing sequence of
 indices $q_j$ such that $y_{p_{q_j}}\to b$.  In other words, if we
 put $n_j=p_{q_j}$ then $y_{n_j}\to b$.  Moreover, the sequence
 $(x_{n_j})$ is a subsequence of $(x_{m_i})$, and $x_{m_i}\to a$ so
 $x_{n_j}\to a$.  It follows that $(x_{n_j},y_{n_j})\to(a,b)$ as
 required. 
\end{proof}

\begin{definition}
 Let $F$ be a subset of $\R^2$.  We say that $F$ is \emph{closed} if
 whenever $(x_n,y_n)$ is a sequence in $F$ that converges to a point
 $(a,b)\in\R^2$, we also have $(a,b)\in F$.
\end{definition}

\begin{example}
 Consider the sets 
 \begin{align*}
  F_0 &= \{(x,y)\in\R^2\st x^2+y^2\leq 1\} \\
  F_1 &= \{(x,y)\in\R^2\st x^2+y^2< 1\}.
 \end{align*}
 Then $F_0$ is closed but $F_1$ is not.  For example, the sequence
 $(x_n,y_n)=(2^{-n},0)$ lies in $F_1$ and converges to the point
 $(a,b)=(1,0)$, which does not lie in $F_1$; so $F_1$ is not closed.
 In general, sets defined using $=$ and $\leq$ will usually be closed,
 but sets defined using $<$ will often not be closed.  However, this
 rule is not always reliable, so you should work from the official
 definition. 
\end{example}

\begin{definition}
 Given a point $u\in\R^2$, we define 
 \[ B_{\ep}(u) = \{v\in\R^2\st \|u-v\|<\ep\}. \]
\end{definition}

\begin{definition}
 Let $U$ be a subset of $\R^2$.  We say that $U$ is \emph{open} if for
 every $u\in U$ there exists $\ep>0$ such that $B_\ep(u)\sse U$.
\end{definition}

\begin{example}
 Consider the sets 
 \begin{align*}
  U_0 &= \{(x,y)\in\R^2\st x>0 \text{ and } y>0\} \\
  U_1 &= \{(x,y)\in\R^2\st x>0 \text{ and } y\geq 0\} \\
  U_2 &= \{(x,y)\in\R^2\st x\geq 0 \text{ and } y\geq 0\}.
 \end{align*}
 Then $U_0$ is open, but $U_1$ and $U_2$ are not.  Indeed, the point
 $u=(1,0)$ is an element of $U_1$, but for every $\ep>0$ we have a
 point $v=(1,-\ep/2)$ which lies in $B_\ep(u)$ but not in $U_1$, so
 $B_\ep(u)\not\sse U_1$.  This proves that $U_1$ is not open, and the
 same example also shows that $U_2$ is not open.  In general, sets
 defined using $\neq$ and $<$ are usually open, but sets defined using
 $\leq$ will often not be open.  However, this rule is not always
 reliable, so you should work from the official definition.
\end{example}

\begin{remark}
 Most subsets of $\R^2$ are neither open nor closed.  For example, the
 set $U_1$ described above is not open, and also it is not closed.
\end{remark}

\begin{proposition}
 Let $F$ and $U$ be two subsets of $\R^2$ that are complements of each
 other, so that 
 \begin{align*}
  U &= \{p\in\R^2\st p\not\in F\} \\
  F &= \{p\in\R^2\st p\not\in U\}.
 \end{align*}
 Then $F$ is closed if and only if $U$ is open.
\end{proposition}
\begin{proof}
 We will prove that $F$ is not closed if and only if $U$ is not open.
 A little thought will convince you that this is equivalent to the
 claimed statement.

 Suppose that $F$ is not closed.  This means that there is a sequence
 of points $v_n=(x_n,y_n)\in F$ and a point $u=(a,b)\not\in F$ such
 that $v_n\to u$.  As $u\not\in F$ we instead have $u\in U$.  Consider
 a number $\ep>0$.  As $v_n\to u$, there exists $N$ such that
 $\|v_N-u\|<\ep$, or in other words $v_N\in B_\ep(u)$.  Now
 $v_N\in F$, so $v_N\not\in U$, so $B_\ep(u)\not\sse U$.  As this
 holds for all $\ep>0$, we see that $U$ is not open.

 Conversely, suppose that $U$ is not open.  This means that there is a
 point $u\in U$ such that there is no $\ep>0$ with $B_\ep(u)\sse U$.
 In particular, $B_{1/n}(u)$ is not contained in $U$, so we can choose
 a point $v_n\in B_{1/n}(u)$ such that $v_n\not\in U$, or equivalently
 $v_n\in F$.  As $v_n\in B_{1/n}(u)$ we have $\|v_n-u\|<1/n$, and this
 implies that $v_n\to u$.  Now the points $v_n$ form a sequence in
 $F$, which converges to the point $u$, which does not lie in $F$.
 This means that $F$ is not closed.
\end{proof}

We now consider a differential equation and its solutions. Let $f$ and
$g$ be continuously differentiable functions on $\R^2$.  For any point
$(x_0,y_0)\in\R^2$, we would like to consider maps $x,y\:\R\to\R^2$
such that $x(0)=x_0$ and $y(0)=y_0$ and
\begin{align*}
 \dot{x}(t) &= f(x(t),y(t)) \\
 \dot{y}(t) &= g(x(t),y(t))
\end{align*}
for all $t\in\R$.  Unfortunately, this is not always possible.  

\begin{example}
 Consider the equations $\dot{x}=x^2$ and $\dot{y}=0$ with $x=y=1$ at
 $t=0$.  There is a solution $(x,y)=(1/(1-t),1)$, but this becomes
 undefined at $t=1$.  There is no solution that is defined and
 differentiable on all of $\R$.
\end{example}

However, the above example is essentially the worst thing that can
happen, as shown by the following result.

\begin{theorem}
 For any point $(x_0,y_0)\in\R^2$, there is a unique pair of
 continuously differentiable functions $x,y\:(\al,\bt)\to\R$ such that
 \begin{itemize}
  \item $\al$ is either $-\infty$ or a negative real number.
  \item $\bt$ is either $+\infty$ or a positive real number.
  \item $x(0)=x_0$ and $y(0)=y_0$
  \item For all $t\in(\al,\bt)$ we have $\dot{x}(t)=f(x(t),y(t))$ and
   $\dot{y}(t)=g(x(t),y(t))$.
  \item Either $\al=-\infty$, or $\al$ is finite and
   $\|(x(t),y(t))\|\to\infty$ as $t\to\al$.
  \item Either $\bt=+\infty$, or $\bt$ is finite and
   $\|(x(t),y(t))\|\to\infty$ as $t\to\bt$.
 \end{itemize}
\end{theorem}

This is a very important theorem, which we have implicitly been using
throughout the course, but we will not prove it.  A proof can be found
in the recommended book by Teschl.

\begin{assumption}
 For simplicity, we will consider only systems for which $\al$ is
 always $-\infty$ and $\bt$ is always $+\infty$, so the solutions are
 defined for all $t\in\R$.
\end{assumption}

\begin{definition}
 We put $\phi(t,(x_0,y_0))=(x(t),y(t))$, where $x(t)$ and $y(t)$ give
 the unique solution as discussed above.
\end{definition}

\begin{theorem}
 The function $\phi$ is continuously differentiable as a map from
 $\R^3$ to $\R^2$.
\end{theorem}
Again, a proof can be found in the book by Teschl.

\begin{assumption}
 From now on, we make the following additional assumptions:
 \begin{itemize}
  \item[(a)] $R$ is an open subset of $\R^2$.
  \item[(b)] $e=(a,b)$ is a point in $R$ where $f(e)=g(e)=0$.
  \item[(c)] $V$ is a continuously differentiable function on $R$.
  \item[(d)] $V(e)=0$, and $V(u)>0$ for all other points $u\in R$.  (In
   other words, $V$ is positive definite.)
  \item[(e)] The function $W=V_xf+V_yg$ satisfies $W(e)=0$, and $W(u)<0$
   for all other points $u\in R$.  (In other words, $W$ is negative
   definite.) 
 \end{itemize}
 This means that $V$ is a \emph{strong Lyapunov function} for $R$ and
 $e$. 
\end{assumption}

\begin{remark}\lbl{rem-decreasing}
 Suppose we have a solution $u(t)=(x(t),y(t))$ to our equations.  We
 then have 
 \begin{align*}
  \frac{d}{dt}V(x(t),y(t))
   &= V_x(x(t),y(t))\dot{x}(t) + V_y(x(t),y(t))\dot{y}(t) \\
   &= V_x(x(t),y(t))f(x(t),y(t)) + V_y(x(t),y(t))g(x(t),y(t)) \\
   &= W(x(t),y(t)) \leq 0,
 \end{align*}
 so $V(x(t),y(t))$ is a weakly decreasing function of $t$.  However,
 this is only valid as long as the curve $u(t)=(x(t),y(t))$ stays
 inside $R$.  Suppose that $t_0<t_1<t_2$ and $u(t_0),u(t_2)\in R$ but
 $u(t_1)\not\in R$.  Then the above argument is not valid when
 $t=t_1$, because $V(u(t_1))$ and $W(u(t_1))$ are undefined, so we
 cannot conclude that $u(t_2)\leq u(t_0)$.
\end{remark}

\begin{definition}\lbl{defn-stable}
 The point $e$ is \emph{asymptotically stable} if for all $\ep>0$
 there exists $\ep'>0$ such that whenever $\|u-e\|<\ep'$, we have
 $\|\phi(t,u)-e\|<\ep$ for all $t\geq 0$, and also $\phi(t,u)\to e$ as
 $t\to\infty$. 
\end{definition}

We will show that whenever there is a Lyapunov function as described
above, the point $e$ is automatically asymptotically stable.

\begin{definition}\lbl{defn-U}
 We choose a number $\ep_0>0$ such that $B_{\ep_0}(e)\sse R$.  (This
 is always possible, because $R$ is assumed to be an open set
 containing $e$.)  When $0<\ep<\ep_0$, we put 
 \begin{align*}
  U_\ep &= B_\ep(e) = \{u\in\R^2\st \|u-e\|<\ep\} \\
  \ov{U}_\ep &=  \{u\in\R^2\st \|u-e\|\leq\ep\},
 \end{align*}
 so $U_\ep$ is open, $\ov{U}_\ep$ is closed, and
 $e\in U_\ep\sse\ov{U}_\ep\sse R$.
\end{definition}

\begin{definition}\lbl{defn-S}
 Given $\dl>0$, we put $S^*_{\dl}=\{u\in R\st V(u)\leq\dl\}$.  This
 contains $e$, and it may split up into several different components.
 Suppose we have two points $u,v\in S^*_\dl$.  We say that they are
 \emph{connected in $S^*_\dl$} if there is a continuous map
 $\gm\:[0,1]\to S^*_\dl$ with $\gm(0)=u$ and $\gm(1)=v$.  We put 
 \[ S_\dl = \{u\in S_\dl^*\st u \text{ is connected to } e
               \text{ in } S_\dl^* \}.
 \]
 Note that $e\in S_\dl\sse S^*_\dl\sse R$.
\end{definition}

\begin{example}\lbl{eg-disc-connected}
 We will show later that for each $\dl>0$ there exists $\ep>0$ such that
 $U_\ep\sse S_\dl^*$.  Every point $u\in U_\ep$ can be connected to
 $e$ by the path $\gm(t)=u+t(e-u)$, which goes in a straight line from
 $u$ to $e$.  This path stays inside $U_\ep$, and $U_\ep\sse S_\dl^*$,
 so $\gm$ stays inside $S_\dl^*$.  This shows that $u$ actually lies
 in $S_\dl$, not just $S_\dl^*$.  We conclude that $U_\ep\sse S_\dl$.
\end{example}

\begin{example}\lbl{eg-flow-connected}
 Suppose that $u\in S^*_\dl\subset R$, and that $v=\phi(t,u)$ for some
 $t>0$.  Suppose also that $\phi(s,u)\in R$ for all $s$ with
 $0\leq s\leq t$.  In other words, there is a flow line running from
 $u$ to $v$, and it stays in $R$.  By Remark~\ref{rem-decreasing}, we
 have $V(v)\leq V(u)\leq\dl$, so $v\in S_\dl^*$.

 Now suppose that $u$ is in $S_\dl$, so there is a path $\gm$ from $e$
 to $u$ in $S_\dl^*$.  The flow line gives a path from $u$ to $v$ in
 $S_\dl^*$, and we can join these two paths together to get a path
 from $e$ to $v$ in $S_\dl^*$.  This shows that $v$ actually lies in
 $S_\dl$.
\end{example}

\begin{proposition}\label{prop-U-S}
 For all $\dl>0$, there exists $\ep$ with $0<\ep<\ep_0$ such that
 $U_\ep\sse S_\dl^*$.
\end{proposition}
\begin{proof}
 We will suppose that there is no such $\ep$, and deduce a
 contradiction. 

 Let $n_0$ be a large integer, big enough that $1/n_0<\ep_0$.  By
 assumption, for $n\geq n_0$ we have $U_{1/n}\not\sse S_\dl^*$.  We
 can therefore choose $u_n\in U_{1/n}\sse R$ with
 $u_n\not\in S_\dl^*$.  As $u_n\not\in S_\dl^*$, we must have
 $V(u_n)>\dl$.  However, as $u_n\in U_{1/n}$ we have $\|u-e\|<1/n$, so
 $u_n\to e$.  As $V$ is continuous on $R$,  it follows that
 $V(u_n)\to V(e)=0$.  As $V(u_n)>\dl$ for all $n$, this is
 impossible.  We therefore have a contradiction, so the Proposition
 must be true after all.
\end{proof}

\begin{corollary}\lbl{cor-U-S}
 For all $\dl>0$, there exists $\ep$ with $0<\ep<\ep_0$ such that
 $U_\ep\sse S_\dl$. 
\end{corollary}
\begin{proof}
 This follows from the Proposition by
 Example~\ref{eg-disc-connected}. 
\end{proof}

\begin{proposition}\lbl{prop-S-U}
 For all $\ep$ with $0<\ep<\ep_0$, there exists $\dl>0$ such that
 $S_\dl\sse U_\ep$.
\end{proposition}
\begin{proof}
 We will suppose that there is no such $\dl$, and deduce a
 contradiction. 

 By assumption, for all $n>0$, the set $S_{1/n}$ is not contained in
 $U_\ep$.  We can therefore choose a point $u_n\in S_{1/n}$ such that
 $u_n\not\in U_\ep$.  As $u_n\not\in U_\ep$, we have
 $\|u-e\|\geq\ep$.  As $u_n\in S_\dl$, we can choose a continuous map
 $\gm\:[0,1]\to S_\dl^*$ with $\gm(0)=e$ and $\gm(1)=u_n$.  Now
 $\|\gm(t)-e\|$ is a continuous function of $t$ that is $0$ when
 $t=0$, and is greater than or equal to $\ep$ when $t=1$.  By the
 Intermediate Value Theorem, there exists $t_0\in [0,1]$ with
 $\|\gm(t_0)-e\|=\ep$.  We put $v_n=\gm(t_0)$, so $\|v_n-e\|=\ep$.
 Moreover, the path $\gm$ joins $e$ to $v_n$ in $S_\dl^*$, so
 $v_n\in S_\dl$.  Because $\|v_n-e\|=\ep$, we have
 $\|v_n\|\leq\|e\|+\ep$ for all $n$, so the sequence $(v_n)$ is
 bounded.  Thus, the Bolzano-Weierstrass theorem says that we can find
 a subsequence $(v_{n_k})$ and a point $w\in\R^2$ such that
 $v_{n_k}\to w$.  It follows that $\|v_{n_k}-e\|\to\|w-e\|$, but
 $\|v_{n_k}-e\|=\ep$, so $\|w-e\|=\ep$.  In particular, we have
 $\|w-e\|<\ep_0$, so $w\in U_{\ep_0}\sse R$.  As $V$ is continuous on
 $R$, it follows that $V(v_{n_k})\to V(w)$.  However,
 $v_n\in S_{1/n}$, so $0\leq V(v_n)\leq 1/n$, so $V(v_n)\to 0$.  This
 means that $w\in R$ with $V(w)=0$, but $V$ is positive definite, so
 $w$ must be equal to $e$.  This means that $v_{n_k}\to e$, but that
 is impossible because $\|v_{n_k}-e\|=\ep>0$ for all $k$.

 Thus, the Proposition must be true after all.
\end{proof}

\begin{proposition}\lbl{prop-invariant}
 Suppose that $0<\ep<\ep_0$, and that $S_\dl\sse U_\ep$.  Then any
 flow line that starts in $S_\dl$ stays in $S_\dl$ for all $t$.
\end{proposition}
\begin{proof}
 Consider a flow line $u(t)=(x(t),y(t))$ such that $u(0)\in S_\dl$.
 Put 
 \[ t_0=\sup\{t\st u([0,t])\sse S_\dl\}. \]
 We want to prove that $t_0=\infty$.  We will suppose that
 $t_0<\infty$, and deduce a contradiction.  For $0\leq t<t_0$ we have
 $u(t)\in S_\dl\sse U_\ep$, so $\|u(t)-e\|<\ep$.  It follows that
 $\|u(t_0)-e\|\leq\ep<\ep_0$.  By continuity, there exists $t_1>t_0$
 such that $\|u(t)-e\|<\ep_0$ for $0\leq t\leq t_1$.  This means that
 the flow line stays within $U_{\ep_0}\sse R$ from $t=0$ to $t=t_1$.
 Thus, Remark~\ref{rem-decreasing} tells us that
 $V(u(t))\leq V(u(0))\leq\dl$ for $0\leq t\leq t_1$, so these points
 $u(t)$ all lie in $S_\dl^*$.  Moreover, all these points are
 connected together in $S_\dl^*$ by the flow line, so they lie in
 $S_\dl$ by Example~\ref{eg-flow-connected}.  Thus,
 $u([0,t_1])\sse S_\dl$.  This is a contradiction, because $t_1>t_0$,
 and $t_0$ was defined to be the supremum of all numbers $t$ such that
 $u([0,t])\sse S_\dl$.

 We must therefore have $t_0=\infty$ after all, which means that
 $u(t)\in S_\dl$ for all $t\geq 0$. 
\end{proof}

\begin{theorem}
 The point $e$ is asymptotically stable.
\end{theorem}
\begin{proof}
 Suppose given $\ep>0$.  Put $\ep_1=\min(\ep_0/2,\ep)$, so
 $0<\ep_1<\ep_0$.  Proposition~\ref{prop-S-U} says that we can choose
 $\dl$ such that $S_\dl\sse U_{\ep_1}$.  Corollary~\ref{cor-U-S}
 says that we can choose $\ep_2$ such that $U_{\ep_2}\sse S_\dl$.
 Consider a flow line $u$ that starts at
 $u(0)\in U_{\ep_2}\sse S_{\dl}$.  By
 Proposition~\ref{prop-invariant}, we have
 $u(t)\in S_\dl\sse U_{\ep_1}\sse B_{\ep}(e)\sse R$ for all
 $t\geq 0$.  In other words, the flow line is completely contained in
 $R$, so the function $V(u(t))$ is defined for all $t$ and is weakly
 decreasing by Remark~\ref{rem-decreasing}.  Any nonnegative, weakly
 decreasing function converges to a limit, so $V(u(t))$ converges to
 some number $V_0$ as $t\to\infty$.  As $0\leq V(u(t))\leq\dl$ for
 $t\geq 0$, we also have $0\leq V_0\leq \dl$.

 Next, as $u(t)\in U_{\ep_1}$ we have $\|u(t)\|\leq \|e\|+\ep_1$ for
 all $t\geq 0$.  This means that the sequence $(u(n))$ is bounded, so
 by the Bolzano-Weierstrass theorem there is a subsequence $(u(n_k))$
 that converges to some point $v\in\R^2$.  Here $\|u(n_k)-e\|<\ep_1$
 for all $k$, so $\|v-e\|\leq\ep_1<\ep_0$, so $v\in R$.  As $V$ is
 continuous on $R$ we have $V(u(n_k))\to V(v)$, but also
 $V(u(t))\to V_0$, so $V(v)=V_0$.  In particular, this means that
 $v\in S_\dl$.

 Next, consider $\phi(s,v)$ for some $s>0$.  These points lie on a
 flow line starting at the point $v\in S_\dl$ when $s=0$, so they are
 still in $S_\dl$ by Proposition~\ref{prop-invariant}.  Note also that
 $\phi(s,u(t))=u(s+t)$, so 
 \[ \phi(s,v) = \lim_{k\to\infty} \phi(s,u(n_k)) = 
     \lim_{k\to\infty} u(s+n_k),
 \]
 so 
 \[ V(\phi(s,v))=\lim_{k\to\infty}V(u(s+n_k)), \]
 but $V(u(t))\to V_0$, so $V(\phi(s,v))=V_0$, which is independent of
 $s$.  This means that $W(\phi(s,v))=\frac{d}{ds}V(\phi(s,v))=0$, but
 $W$ is negative definite, so $\phi(s,v)=e$.  In particular, we can
 take $s=0$, and we get $v=e$, so $V_0=V(v)=V(e)=0$.  This means
 that $V(u(t))\to 0$ as $t\to\infty$.

 Finally, suppose we have another small number $\ep'>0$.  By
 Proposition~\ref{prop-S-U}, there exists $\dl'>0$ such that
 $S_{\dl'}\sse U_{\ep'}$.  As $V(u(t))\to 0$, we can find $T$ such
 that $V(u(t))<\dl'$ for $t\geq T$.  This means that when $t\geq T$ we
 have $u(t)\in S_{\dl'}\sse U_{\ep'}$, so $\|u(t)-e\|<\ep'$.  This
 proves that $u(t)\to e$, as required.  Thus, $e$ is asymptotically
 stable. 
\end{proof}


\end{document}

