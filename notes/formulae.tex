\documentclass[reqno]{amsart}
%\usepackage{hyperref}
\usepackage{fullpage}
\usepackage{amsrefs}
\usepackage{verbatim}
\usepackage{tikz}
\usepackage{graphicx}

\input{../macros/macros}

\begin{document}

\title{MAS290 Methods for Differential Equations --- Formulae}

\maketitle

You should learn and remember all the formulae on this sheet.

\section*{Matrices}

For a matrix $A=\bbm a&b\\ c&d\ebm$, the trace is $\tau=a+d$ and the
determinant is $\dl=ad-bc$.  The eigenvalues are
$\lm_1=(\tau-\sqrt{\tau^2-4\dl})/2$ and
$\lm_2=(\tau+\sqrt{\tau^2-4\dl})/2$, and we also have
$\lm_1+\lm_2=\tau$ and $\lm_1\lm_2=\dl$.  The corresponding linear system
can be classified as follows.
\begin{itemize}
 \item If $\dl<0$ then $\lm_1$ and $\lm_2$ are real with
  $\lm_1<0<\lm_2$, and we have a saddle.
 \item If $\dl>0$ and $\tau>0$ and $\tau^2-4\dl>0$ then  $\lm_1$ and
  $\lm_2$ are real with $0<\lm_1<\lm_2$, and we have an unstable
  node. 
 \item If $\dl>0$ and $\tau>0$ and $\tau^2-4\dl<0$ then  $\lm_1$ and
  $\lm_2$ are complex with $0<\text{Re}(\lm_1)=\text{Re}(\lm_2)$, and
  we have an unstable focus.  The rotation is anticlockwise if
  $b<0<c$, and clockwise if $c<0<b$.
 \item If $\dl>0$ and $\tau=0$ then we have a centre.  The eigenvalues
  are $\pm i\om$, where $\om=\sqrt{|\dl|}$.  The rotation is
  anticlockwise if $b<0<c$, and clockwise if $c<0<b$.
 \item If $\dl>0$ and $\tau<0$ and $\tau^2-4\dl<0$ then  $\lm_1$ and
  $\lm_2$ are complex with $\text{Re}(\lm_1)=\text{Re}(\lm_2)<$, and
  we have a stable focus.  The rotation is anticlockwise if
  $b<0<c$, and clockwise if $c<0<b$.
 \item If $\dl>0$ and $\tau<0$ and $\tau^2-4\dl>0$ then  $\lm_1$ and
  $\lm_2$ are real with $\lm_1<\lm_2<0$, and we have a stable
  node.
 \item Cases where $\dl=0$ or $\tau^2-4\dl=0$ will not be discussed
  here. 
\end{itemize}

\section*{Fundamental solutions}

The fundamental solution for a matrix $A$ is a matrix $P$ depending on
$t$ with $\dot{P}=AP$ and $P=I$ when $t=0$.

\begin{itemize}
 \item Suppose that there are eigenvalues $\lm_1$ and $\lm_2$ with
  corresponding eigenvectors $v_1$ and $v_2$ that are linearly
  independent.  Put 
  \[ V = \barmat{v_1}{v_2} \qquad
     D = \bbm \lm_1 & 0 \\ 0 & \lm_2 \ebm \qquad 
     E = \bbm e^{\lm_1t} & 0 \\ 0 & e^{\lm_2t} \ebm.
  \]
  Then $A=VDV^{-1}$ and $P=VEV^{-1}$.
 \item If $\lm_1\neq\lm_2$ then we also have
  \[ P=(\lm_2-\lm_1)^{-1}((\lm_2e^{\lm_1t}-\lm_1e^{\lm_2t})I +
                        (e^{\lm_2t}-e^{\lm_1t})A)
  \]
 \item If $A$ has complex eigenvalues $\lm\pm i\om$ (with $\om\neq 0$)
  then the above formula can also be written as 
  \[ P=e^{\lm t}(\cos(\om t)I+\om^{-1}\sin(\om t)(A-\lm I)) \]
 \item If $A$ has only one eigenvalue $\lm$, then we instead have
  \[ P = e^{\lm t}(I + t(A-\lm I)). \]
 \item In all cases we have $\det(P)=e^{\tau t}$, where
  $\tau=\trc(A)=\lm_1+\lm_2$. 
\end{itemize}

\section*{Definiteness of quadratic functions}

Consider a quadratic function $Q=ax^2+2bxy+cy^2$.
\begin{itemize}
  \item If $ac-b^2>0$ and $a,c>0$ then $Q$ is positive definite.
  \item If $ac-b^2>0$ and $a,c<0$ then $Q$ is negative definite.
  \item If $ac-b^2\leq 0$ then $Q$ is neither positive
   definite nor negative definite.
\end{itemize}

\section*{Constant coefficients}

Consider an equation $Ay''+By'+Cy=0$, where $A$, $B$ and $C$ are
constant with $A\neq 0$.  Let $\lm_1$ and $\lm_2$ be the roots of the
auxiliary polynomial $At^2+Bt+C$.
\begin{itemize}
 \item If $\lm_1\neq\lm_2$ then the general solution is
  $y=Pe^{\lm_1x}+Qe^{\lm_2x}$ with $P$ and $Q$ constant.
 \item If $\lm_1,\lm_2=\lm\pm i\om$ (with $\om\neq 0$) then the
  general solution can also be given in the form $y=e^{\lm
   x}(M\cos(\om x)+N\sin(\om x))$ with $M$ and $N$ constant.
 \item If there is only one root $\lm$, then the general solution is
  $y=e^{\lm x}(P+Qx)$.
\end{itemize}

\section*{Reduction of order}

Suppose that $y$ satisfies $Ay''+By'+Cy=0$.  Put $v=\int B/A\,dx$ and
$u=\int y^{-2}e^{-v}\,dx$ and $z=uy$.  Then we also have
$Az''+Bz'+Cz=0$. 

\section*{Sturm-Liouville form}

Consider an operator $L(y)=Ay''+By'+Cy$.  Then we also have
$L(y)=((py')'+qy)/r$, where
\[ v = \int B/A\,dx
   \qquad p = e^v  
   \qquad q = pC/A
   \qquad r = p/A.
\]

\section*{Normal form}

Consider an operator $L(y)=y''+Py'+Qy$.  Put 
\[ v = \int P\,dx 
   \qquad m = e^{-v/2}
   \qquad R = Q - \half P' - \tfrac{1}{4}P^2
   \qquad z = y/m.
\]
Then $y''+Py'+Qy=0$ if and only if $z''+Rz=0$.

\end{document}
