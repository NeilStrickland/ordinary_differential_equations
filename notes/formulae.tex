\documentclass[reqno]{amsart}
%\usepackage{hyperref}
\usepackage{fullpage}
\usepackage{amsrefs}
\usepackage{verbatim}
\usepackage{tikz}
\usepackage{graphicx}

\definecolor{olivegreen}{cmyk}{0.64,0,0.95,0.40}
\definecolor{rawsienna}{cmyk}{0,0.72,1,0.45}
\definecolor{lightgreen}{rgb}{0.85,1.0,0.85}

\newcommand{\GREENYELLOW}[1]{{\color{greenyellow}#1}}
\newcommand{\YELLOW}[1]{{\color{yellow}#1}}
\newcommand{\YLW}[1]{{\color{yellow}#1}}
\newcommand{\GOLDENROD}[1]{{\color{goldenrod}#1}}
\newcommand{\DANDELION}[1]{{\color{dandelion}#1}}
\newcommand{\APRICOT}[1]{{\color{apricot}#1}}
\newcommand{\PEACH}[1]{{\color{peach}#1}}
\newcommand{\MELON}[1]{{\color{melon}#1}}
\newcommand{\YELLOWORANGE}[1]{{\color{yelloworange}#1}}
\newcommand{\ORANGE}[1]{{\color{orange}#1}}
\newcommand{\BURNTORANGE}[1]{{\color{burntorange}#1}}
\newcommand{\BITTERSWEET}[1]{{\color{bittersweet}#1}}
\newcommand{\REDORANGE}[1]{{\color{redorange}#1}}
\newcommand{\MAHOGANY}[1]{{\color{mahogany}#1}}
\newcommand{\MAROON}[1]{{\color{maroon}#1}}
\newcommand{\BRICKRED}[1]{{\color{brickred}#1}}
\newcommand{\RED}[1]{{\color{red}#1}}
\newcommand{\ORANGERED}[1]{{\color{orangered}#1}}
\newcommand{\RUBINERED}[1]{{\color{rubinered}#1}}
\newcommand{\WILDSTRAWBERRY}[1]{{\color{wildstrawberry}#1}}
\newcommand{\SALMON}[1]{{\color{salmon}#1}}
\newcommand{\CARNATIONPINK}[1]{{\color{carnationpink}#1}}
\newcommand{\MAGENTA}[1]{{\color{magenta}#1}}
\newcommand{\VIOLETRED}[1]{{\color{violetred}#1}}
\newcommand{\RHODAMINE}[1]{{\color{rhodamine}#1}}
\newcommand{\MULBERRY}[1]{{\color{mulberry}#1}}
\newcommand{\REDVIOLET}[1]{{\color{redviolet}#1}}
\newcommand{\FUCHSIA}[1]{{\color{fuchsia}#1}}
\newcommand{\LAVENDER}[1]{{\color{lavender}#1}}
\newcommand{\THISTLE}[1]{{\color{thistle}#1}}
\newcommand{\ORCHID}[1]{{\color{orchid}#1}}
\newcommand{\DARKORCHID}[1]{{\color{darkorchid}#1}}
\newcommand{\PURPLE}[1]{{\color{purple}#1}}
\newcommand{\PLUM}[1]{{\color{plum}#1}}
\newcommand{\VIOLET}[1]{{\color{violet}#1}}
\newcommand{\ROYALPURPLE}[1]{{\color{royalpurple}#1}}
\newcommand{\BLUEVIOLET}[1]{{\color{blueviolet}#1}}
\newcommand{\PERIWINKLE}[1]{{\color{periwinkle}#1}}
\newcommand{\CADETBLUE}[1]{{\color{cadetblue}#1}}
\newcommand{\CORNFLOWERBLUE}[1]{{\color{cornflowerblue}#1}}
\newcommand{\MIDNIGHTBLUE}[1]{{\color{midnightblue}#1}}
\newcommand{\NAVYBLUE}[1]{{\color{navyblue}#1}}
\newcommand{\ROYALBLUE}[1]{{\color{royalblue}#1}}
\newcommand{\BLU}[1]{{\color{blue}#1}}
\newcommand{\BLUE}[1]{{\color{blue}#1}}
\newcommand{\CERULEAN}[1]{{\color{cerulean}#1}}
\newcommand{\CYAN}[1]{{\color{cyan}#1}}
\newcommand{\PROCESSBLUE}[1]{{\color{processblue}#1}}
\newcommand{\SKYBLUE}[1]{{\color{skyblue}#1}}
\newcommand{\TURQUOISE}[1]{{\color{turquoise}#1}}
\newcommand{\TEALBLUE}[1]{{\color{tealblue}#1}}
\newcommand{\AQUAMARINE}[1]{{\color{aquamarine}#1}}
\newcommand{\BLUEGREEN}[1]{{\color{bluegreen}#1}}
\newcommand{\EMERALD}[1]{{\color{emerald}#1}}
\newcommand{\JUNGLEGREEN}[1]{{\color{junglegreen}#1}}
\newcommand{\SEAGREEN}[1]{{\color{seagreen}#1}}
\newcommand{\GREEN}[1]{{\color{green}#1}}
\newcommand{\FORESTGREEN}[1]{{\color{forestgreen}#1}}
\newcommand{\PINEGREEN}[1]{{\color{pinegreen}#1}}
\newcommand{\LIMEGREEN}[1]{{\color{limegreen}#1}}
\newcommand{\YELLOWGREEN}[1]{{\color{yellowgreen}#1}}
\newcommand{\SPRINGGREEN}[1]{{\color{springgreen}#1}}
\newcommand{\OLIVEGREEN}[1]{{\color{olivegreen}#1}}
\newcommand{\OLG}[1]{{\color{olivegreen}#1}}
\newcommand{\RAWSIENNA}[1]{{\color{rawsienna}#1}}
\newcommand{\SEPIA}[1]{{\color{sepia}#1}}
\newcommand{\BROWN}[1]{{\color{brown}#1}}
\newcommand{\TAN}[1]{{\color{tan}#1}}
\newcommand{\GRAY}[1]{{\color{gray}#1}}
\newcommand{\LGRAY}[1]{{\color{gray!40}#1}}
\newcommand{\WHITE}[1]{{\color{white}#1}}
\newcommand{\BLACK}[1]{{\color{black}#1}}

\newcommand{\bbm}       {\left[\begin{matrix}}
\newcommand{\ebm}       {\end{matrix}\right]}
\newcommand{\bsm}       {\left[\begin{smallmatrix}}
\newcommand{\esm}       {\end{smallmatrix}\right]}
\newcommand{\bpm}       {\begin{pmatrix}}
\newcommand{\epm}       {\end{pmatrix}}
\newcommand{\bcf}[2]{\left(\begin{array}{c}{#1}\\{#2}\end{array}\right)}

\newcommand{\adj}       {\operatorname{adj}}
\newcommand{\ann}       {\operatorname{ann}}
\newcommand{\diag}      {\operatorname{diag}}
\newcommand{\img}       {\operatorname{img}}
\newcommand{\rnk}       {\operatorname{rank}}
\newcommand{\sgn}       {\operatorname{sgn}}
\newcommand{\spn}       {\operatorname{span}}
\newcommand{\trc}       {\operatorname{trace}}

\newcommand{\pp}{\hphantom{+}}
\newcommand{\tm}{\times}
\newcommand{\sse}{\subseteq}
\newcommand{\st}{\;|\;}
\newcommand{\sm}{\setminus}
\newcommand{\iffa}      {\Leftrightarrow}
\newcommand{\xra}{\xrightarrow}
\newcommand{\xla}{\xleftarrow}

\newcommand{\half}{\tfrac{1}{2}}

\newcommand{\N}         {{\mathbb{N}}}
\newcommand{\Z}         {{\mathbb{Z}}}
\newcommand{\Q}         {{\mathbb{Q}}}
\newcommand{\R}         {{\mathbb{R}}}
\newcommand{\C}         {{\mathbb{C}}}

\newcommand{\va}        {\mathbf{a}}
\newcommand{\vb}        {\mathbf{b}}
\newcommand{\vc}        {\mathbf{c}}
\newcommand{\vd}        {\mathbf{d}}
\newcommand{\ve}        {\mathbf{e}}
\newcommand{\vf}        {\mathbf{f}}
\newcommand{\vg}        {\mathbf{g}}
\newcommand{\vh}        {\mathbf{h}}
\newcommand{\vi}        {\mathbf{i}}
\newcommand{\vj}        {\mathbf{j}}
\newcommand{\vk}        {\mathbf{k}}
\newcommand{\vl}        {\mathbf{l}}
\newcommand{\vm}        {\mathbf{m}}
\newcommand{\vn}        {\mathbf{n}}
\newcommand{\vo}        {\mathbf{o}}
\newcommand{\vp}        {\mathbf{p}}
\newcommand{\vq}        {\mathbf{q}}
\newcommand{\vr}        {\mathbf{r}}
\newcommand{\vs}        {\mathbf{s}}
\newcommand{\vt}        {\mathbf{t}}
\newcommand{\vu}        {\mathbf{u}}
\newcommand{\vv}        {\mathbf{v}}
\newcommand{\vw}        {\mathbf{w}}
\newcommand{\vx}        {\mathbf{x}}
\newcommand{\vy}        {\mathbf{y}}
\newcommand{\vz}        {\mathbf{z}}

\newcommand{\vA}        {\mathbf{A}}
\newcommand{\vB}        {\mathbf{B}}
\newcommand{\vC}        {\mathbf{C}}
\newcommand{\vD}        {\mathbf{D}}
\newcommand{\vE}        {\mathbf{E}}
\newcommand{\vF}        {\mathbf{F}}
\newcommand{\vG}        {\mathbf{G}}
\newcommand{\vH}        {\mathbf{H}}
\newcommand{\vI}        {\mathbf{I}}
\newcommand{\vJ}        {\mathbf{J}}
\newcommand{\vK}        {\mathbf{K}}
\newcommand{\vL}        {\mathbf{L}}
\newcommand{\vM}        {\mathbf{M}}
\newcommand{\vN}        {\mathbf{N}}
\newcommand{\vO}        {\mathbf{O}}
\newcommand{\vP}        {\mathbf{P}}
\newcommand{\vQ}        {\mathbf{Q}}
\newcommand{\vR}        {\mathbf{R}}
\newcommand{\vS}        {\mathbf{S}}
\newcommand{\vT}        {\mathbf{T}}
\newcommand{\vU}        {\mathbf{U}}
\newcommand{\vV}        {\mathbf{V}}
\newcommand{\vW}        {\mathbf{W}}
\newcommand{\vX}        {\mathbf{X}}
\newcommand{\vY}        {\mathbf{Y}}
\newcommand{\vZ}        {\mathbf{Z}}

\newcommand{\al}        {\alpha}
\newcommand{\bt}        {\beta} 
\newcommand{\gm}        {\gamma}
\newcommand{\dl}        {\delta}
\newcommand{\ep}        {\epsilon}
\newcommand{\zt}        {\zeta}
\newcommand{\et}        {\eta}
\newcommand{\tht}       {\theta}
\newcommand{\io}        {\iota}
\newcommand{\kp}        {\kappa}
\newcommand{\lm}        {\lambda}
\newcommand{\ph}        {\phi}
\newcommand{\ch}        {\chi}
\newcommand{\ps}        {\psi}
\newcommand{\rh}        {\rho}
\newcommand{\sg}        {\sigma}
\newcommand{\om}        {\omega}

\newcommand{\Gm}        {\Gamma}
\newcommand{\Dl}        {\Delta}

\newcommand{\CA}        {\mathcal{A}}
\newcommand{\CB}        {\mathcal{B}}
\newcommand{\CC}        {\mathcal{C}}
\newcommand{\CD}        {\mathcal{D}}
\newcommand{\CE}        {\mathcal{E}}
\newcommand{\CF}        {\mathcal{F}}
\newcommand{\CG}        {\mathcal{G}}
\newcommand{\CH}        {\mathcal{H}}
\newcommand{\CI}        {\mathcal{I}}
\newcommand{\CJ}        {\mathcal{J}}
\newcommand{\CK}        {\mathcal{K}}
\newcommand{\CL}        {\mathcal{L}}
\newcommand{\CM}        {\mathcal{M}}
\newcommand{\CN}        {\mathcal{N}}
\newcommand{\CO}        {\mathcal{O}}
\newcommand{\CP}        {\mathcal{P}}
\newcommand{\CQ}        {\mathcal{Q}}
\newcommand{\CR}        {\mathcal{R}}
\newcommand{\CS}        {\mathcal{S}}
\newcommand{\CT}        {\mathcal{T}}
\newcommand{\CU}        {\mathcal{U}}
\newcommand{\CV}        {\mathcal{V}}
\newcommand{\CW}        {\mathcal{W}}
\newcommand{\CX}        {\mathcal{X}}
\newcommand{\CY}        {\mathcal{Y}}
\newcommand{\CZ}        {\mathcal{Z}}


\newcommand{\ov}        {\overline}
\newcommand{\ip}[1]     {\langle #1\rangle}
\renewcommand{\ss}      {\scriptstyle}

\renewcommand{\:}       {\colon}

\newcommand{\barmat}[2]{\left[\begin{array}{c|c}\!\!\raisebox{0pt}[0.45cm][0.35cm]{$#1$} & \raisebox{0pt}[0.45cm][0.35cm]{$#2$}\!\!\end{array}\right]}

\newcommand{\eqpair}[4]{\begin{array}{rl} #1 &= #2 \\ #3 &= #4\end{array}}

\newcommand{\han}[1]{\begin{CJK*}{UTF8}{zhsong}\BLUE{#1}\end{CJK*}}
\newcommand{\bhan}[1]{(\begin{CJK*}{UTF8}{zhsong}\BLUE{#1}\end{CJK*})}

\newcommand{\EMPH}[1]{\emph{\RED{#1}}}
\newcommand{\DEFN}[1]{\emph{\PURPLE{#1}}}
\newcommand{\VEC}[1]    {\mathbf{#1}}

\newcommand{\ghost}{{\tiny $\color[rgb]{1,1,1}.$}}

\newcommand{\reminderbar}{\par\medskip\par\hrule\par\medskip\par}

\newcommand{\uc}{\uncover}

\newcommand{\bbox}[1]{
\[ \mbox{\begin{tikzpicture}%
   \draw(0,0) node[draw,thick,olivegreen,rectangle] {\color{black} #1};%
  \end{tikzpicture}} \]
}

\newcommand{\cbox}[1]{
\begin{center}\begin{tikzpicture}%
   \draw(0,0) node[draw,thick,olivegreen,rectangle] {\color{black} #1};%
\end{tikzpicture}\end{center}
}



\begin{document}

\title{MAS290 Methods for Differential Equations --- Formulae}

\maketitle

You should learn and remember all the formulae on this sheet.

\section*{Matrices}

For a matrix $A=\bbm a&b\\ c&d\ebm$, the trace is $\tau=a+d$ and the
determinant is $\dl=ad-bc$.  The eigenvalues are
$\lm_1=(\tau-\sqrt{\tau^2-4\dl})/2$ and
$\lm_2=(\tau+\sqrt{\tau^2-4\dl})/2$, and we also have
$\lm_1+\lm_2=\tau$ and $\lm_1\lm_2=\dl$.  The corresponding linear system
can be classified as follows.
\begin{itemize}
 \item If $\dl<0$ then $\lm_1$ and $\lm_2$ are real with
  $\lm_1<0<\lm_2$, and we have a saddle.
 \item If $\dl>0$ and $\tau>0$ and $\tau^2-4\dl>0$ then  $\lm_1$ and
  $\lm_2$ are real with $0<\lm_1<\lm_2$, and we have an unstable
  node. 
 \item If $\dl>0$ and $\tau>0$ and $\tau^2-4\dl<0$ then  $\lm_1$ and
  $\lm_2$ are complex with $0<\text{Re}(\lm_1)=\text{Re}(\lm_2)$, and
  we have an unstable focus.  The rotation is anticlockwise if
  $b<0<c$, and clockwise if $c<0<b$.
 \item If $\dl>0$ and $\tau=0$ then we have a centre.  The eigenvalues
  are $\pm i\om$, where $\om=\sqrt{|\dl|}$.  The rotation is
  anticlockwise if $b<0<c$, and clockwise if $c<0<b$.
 \item If $\dl>0$ and $\tau<0$ and $\tau^2-4\dl<0$ then  $\lm_1$ and
  $\lm_2$ are complex with $\text{Re}(\lm_1)=\text{Re}(\lm_2)<$, and
  we have a stable focus.  The rotation is anticlockwise if
  $b<0<c$, and clockwise if $c<0<b$.
 \item If $\dl>0$ and $\tau<0$ and $\tau^2-4\dl>0$ then  $\lm_1$ and
  $\lm_2$ are real with $\lm_1<\lm_2<0$, and we have a stable
  node.
 \item Cases where $\dl=0$ or $\tau^2-4\dl=0$ will not be discussed
  here. 
\end{itemize}

\section*{Fundamental solutions}

The fundamental solution for a matrix $A$ is a matrix $P$ depending on
$t$ with $\dot{P}=AP$ and $P=I$ when $t=0$.

\begin{itemize}
 \item Suppose that there are eigenvalues $\lm_1$ and $\lm_2$ with
  corresponding eigenvectors $v_1$ and $v_2$ that are linearly
  independent.  Put 
  \[ V = \barmat{v_1}{v_2} \qquad
     D = \bbm \lm_1 & 0 \\ 0 & \lm_2 \ebm \qquad 
     E = \bbm e^{\lm_1t} & 0 \\ 0 & e^{\lm_2t} \ebm.
  \]
  Then $A=VDV^{-1}$ and $P=VEV^{-1}$.
 \item If $\lm_1\neq\lm_2$ then we also have
  \[ P=(\lm_2-\lm_1)^{-1}((\lm_2e^{\lm_1t}-\lm_1e^{\lm_2t})I +
                        (e^{\lm_2t}-e^{\lm_1t})A)
  \]
 \item If $A$ has complex eigenvalues $\lm\pm i\om$ (with $\om\neq 0$)
  then the above formula can also be written as 
  \[ P=e^{\lm t}(\cos(\om t)I+\om^{-1}\sin(\om t)(A-\lm I)) \]
 \item If $A$ has only one eigenvalue $\lm$, then we instead have
  \[ P = e^{\lm t}(I + t(A-\lm I)). \]
 \item In all cases we have $\det(P)=e^{\tau t}$, where
  $\tau=\trc(A)=\lm_1+\lm_2$. 
\end{itemize}

\section*{Definiteness of quadratic functions}

Consider a quadratic function $Q=ax^2+2bxy+cy^2$.
\begin{itemize}
  \item If $ac-b^2>0$ and $a,c>0$ then $Q$ is positive definite.
  \item If $ac-b^2>0$ and $a,c<0$ then $Q$ is negative definite.
  \item If $ac-b^2\leq 0$ then $Q$ is neither positive
   definite nor negative definite.
\end{itemize}

\section*{Constant coefficients}

Consider an equation $Ay''+By'+Cy=0$, where $A$, $B$ and $C$ are
constant with $A\neq 0$.  Let $\lm_1$ and $\lm_2$ be the roots of the
auxiliary polynomial $At^2+Bt+C$.
\begin{itemize}
 \item If $\lm_1\neq\lm_2$ then the general solution is
  $y=Pe^{\lm_1x}+Qe^{\lm_2x}$ with $P$ and $Q$ constant.
 \item If $\lm_1,\lm_2=\lm\pm i\om$ (with $\om\neq 0$) then the
  general solution can also be given in the form $y=e^{\lm
   x}(M\cos(\om x)+N\sin(\om x))$ with $M$ and $N$ constant.
 \item If there is only one root $\lm$, then the general solution is
  $y=e^{\lm x}(P+Qx)$.
\end{itemize}

\section*{Reduction of order}

Suppose that $y$ satisfies $Ay''+By'+Cy=0$.  Put $v=\int B/A\,dx$ and
$u=\int y^{-2}e^{-v}\,dx$ and $z=uy$.  Then we also have
$Az''+Bz'+Cz=0$. 

\section*{Sturm-Liouville form}

Consider an operator $L(y)=Ay''+By'+Cy$.  Then we also have
$L(y)=((py')'+qy)/r$, where
\[ v = \int B/A\,dx
   \qquad p = e^v  
   \qquad q = pC/A
   \qquad r = p/A.
\]

\section*{Normal form}

Consider an operator $L(y)=y''+Py'+Qy$.  Put 
\[ v = \int P\,dx 
   \qquad m = e^{-v/2}
   \qquad R = Q - \half P' - \tfrac{1}{4}P^2
   \qquad z = y/m.
\]
Then $y''+Py'+Qy=0$ if and only if $z''+Rz=0$.

\end{document}
