\documentclass{amsart}
\usepackage{hyperref}
\usepackage{fullpage}
\usepackage{amsrefs}

\usepackage[matrix,arrow]{xy}
\usepackage{xypdf}
\newdir{ >}{{}*!/-9pt/\dir{>}}

\newcommand{\Map}       {\operatorname{Map}}

\newcommand{\ann}       {\operatorname{ann}}

\newcommand{\Z}         {{\mathbb{Z}}}
\newcommand{\C}         {{\mathbb{C}}}

\newcommand{\al}        {\alpha}
\newcommand{\bt}        {\beta} 
\newcommand{\dl}        {\delta}

\newcommand{\Gm}        {\Gamma}
\newcommand{\Dl}        {\Delta}

\newcommand{\sm}        {\setminus}
\newcommand{\st}        {\;|\;}
\newcommand{\tE}	{\widetilde{E}}
\newcommand{\tm}        {\times}
\renewcommand{\:}{\colon}

\newtheorem{theorem}{Theorem}[section]
\newtheorem{conjecture}[theorem]{Conjecture}
\newtheorem{lemma}[theorem]{Lemma}
\newtheorem{proposition}[theorem]{Proposition}
\newtheorem{corollary}[theorem]{Corollary}
\theoremstyle{definition}
\newtheorem{remark}[theorem]{Remark}
\newtheorem{definition}[theorem]{Definition}
\newtheorem{example}[theorem]{Example}
\newtheorem{construction}[theorem]{Construction}

\newtheorem{notation}{Notation}
\renewcommand{\thenotation}{} % make the notation environment unnumbered

%\numberwithin{equation}{subsection}

\begin{document}
\title{Meijer $G$-functions}
\author{N.~P.~Strickland}

\maketitle 

\begin{definition}
 We write $W$ for the Weyl algebra generated over $\C$ by $z$ and
 $\partial$ subject to the relation $[\partial,z]=1$.  We give this a
 grading with $|z|=1$ and $|\partial|=-1$.  We put
 $\Dl=z\,\partial\in W_0$. 
\end{definition}

\begin{remark}
 It is not hard to see that $\{\Dl^k\st k\geq 0\}$ and
 $\{z^k\partial^k\st k\geq 0\}$ are both bases for $W_0$ over $\C$.
\end{remark}

\begin{definition}
 We put $M=W_1+W_0$, which is a bimodule for $W_0$.  We call the
 elements of $M$ \emph{Meijer operators}.  Any such operator can be
 written in the form $L=z\,f(\Dl)-g(\Dl)$ for some polynomials $f$ and
 $g$.  The \emph{bidegree} of $L$ is the pair $(\deg(f),\deg(g))$
 (with the convention $\deg(0)=-\infty$). 
\end{definition}

We will study the sets $\ann(L,U)=\{u\in U\st Lu=0\}$ for various
$W$-modules $U$:
\begin{definition}\leavevmode
 \begin{itemize}
  \item[(a)] We write $H$ for the space of holomorphic functions
   $u(z)$ on $\C^\tm$, with $\partial$ acting as differentiation and
   $z$ acting as multiplication by the identity function.  We call
   this the \emph{holomorphic module}.
  \item[(b)] We write $S$ for the space of doubly infinite sequences
   $(a_k)_{k\in\Z}$ that are rapidly decreasing in the sense that 
   $|k^Na_k|\to 0$ as $|k|\to\infty$ for all $N\geq 0$.  This can be
   regarded as a $W$-module by the rules $(\partial a)_k=(k+1)a_{k+1}$
   and $(za)_k=a_{k-1}$ (so $(\Dl a)_k=k\,a_k$).  We also write $S_0$
   for the subset of sequences where $a_k=0$ for $|k|\gg 0$.  We call
   this the \emph{series module}.
  \item[(c)] We write $E$ for the space of holomorphic functions
   $m(t)$ on $\C$, with $\partial$ acting as $e^{-t}\frac{d}{dt}$ and
   $z$ as multiplication by $e^t$.  We call this the \emph{exponential
   module}.
  \item[(d)] We write $F$ for the space of meromorphic functions
   $v(s)$ on $\C$, with $\Dl$ acting as multiplication by $s$,
   and $z$ acting as the shift operator $(zv)(s)=v(s-1)$.  We call
   this the \emph{Mellin module}.
 \end{itemize}
\end{definition}

\begin{remark}
 The exponential module is useful because $\exp\:\C\to\C^\tm$ is a
 universal cover and it turns out that this is sufficient to handle
 all monodromy issues for operators of bidegree $(p,q)$ with
 $p\neq q$.  If $p=q$ then certain relevant functions will have a pole
 at $z=1$ as well as $z\in\{0,\infty\}$ so we need to consider the
 universal cover of $\C\sm\{0,1\}$ by the elliptic modular function
 instead.  We will return to this later.
\end{remark}

\begin{definition}
 We define $\tau\:E\to E$ by $(\tau m)(t)=m(t+2\pi i)$.  For
 $\al\in\C^\tm$ we put 
 \begin{align*}
  E_\al &= \ker(\tau-\al) =
            \{m\in E\st m(t+2\pi i)=\al\,m(t)\text{ for all } t\}\\
  \tE_\al &= \bigcup_{n\geq 0}\ker((\tau-\al)^n)
           = \C[t].E_\al
 \end{align*}
\end{definition}

We can consider various homomorphisms between the above modules.
\begin{itemize}
 \item[(a)] Taylor expansion gives an injective homomorphism
  $\tau\:H\to S$.  In fact, it is well-known that the Fourier
  transform gives an isomorphism from $S$ to the space of smooth
  functions on the circle, and this converts $\tau$ to the obvious
  restriction map.
 \item[(b)] Identifying $z$ with $e^t$ gives an isomorphism between
  $H$ and $E_0<E$.
 \item[(c)] Given a function $v(s)\in F$ we can choose a contour $C$
  in the Riemann sphere and attempt to define
  $u(z)=\oint_C v(s)\,z^s\,ds$, but this can fail in various ways to
  be well-defined.  This construction should give a homomorphism
  between certain groups related to $H$ and $F$, called the
  \emph{Mellin transform}.  However, I  am not yet sure of the best
  formulation for this.   
\end{itemize}

\begin{proposition}
 If $L$ has bidegree $(p,q)$ with $p<q$ then $\ann(L,E)$ has dimension
 $q$ over $\C$.
\end{proposition}
\begin{proof}
 We have $L=zF-G$, where $F\sum_{k=0}^pa_kz^k\partial^k$ and
 $G=\sum_{k=0}^qb_kz^k\partial^k$ say with $a_p,b_q\neq 0$.  This
 means that $L$ acts on $E$ as the operator
 \[ e^t \sum_{k=0}^p a_k\frac{d^k}{dt^k} - 
              \sum_{k=0}^q b_k\frac{d^k}{dt^k}.
 \]
 This is $-b_q$ times a monic polynomial of degree $q$ in
 $\frac{d}{dt}$, with holomorphic coefficients.  The standard
 Frobenius method now shows that for any $t_0$, the kernel of $L$ on
 holomorphic germs at $t_0$ has dimension $q$.  The spaces of local
 solutions form a vector bundle with flat connection over the simply
 connected space $\C$, so the evident map from global solutions to
 germs at $0$ is an isomorphism.
\end{proof}

\begin{corollary}
 If $L$ has bidegree $(p,q)$ with $p>q$ then $\ann(L,E)$ has dimension
 $p$ over $\C$.
\end{corollary}
\begin{proof}
 If $L=zf(\Dl)-g(\Dl)$, put $L^*=zg(-\Dl)-f(-\Dl)$.  The proposition
 shows that $\ann(L^*,E)$ has dimension $p$, and one can check that
 composition with $t\mapsto -t$ gives an isomorphism
 $\ann(L,E)\simeq\ann(L^*,E)$. 
\end{proof}

\begin{corollary}
 If $L$ has bidegree $(p,q)$ with $p\neq q$ then
 $\ann(L,E)=\bigoplus_{\al\neq 0}(\ann(L,\tE_\al))$. 
\end{corollary}
\begin{proof}
 It is not hard to see that $\ann(L,E)$ is preserved by $\tau$.  As
 $\ann(L,E)$ is also finite-dimensional, it must split as a direct sum
 of its generalised eigenspaces.  Note also that $\tau$ is invertible,
 so all eigenvalues are nonzero.  The claim is clear from this.
\end{proof}

We now study the spaces $K=\ann(L,S)$ and $K_0=K\cap S_0$, where again
$L=zf(\Dl)-g(\Dl)$ has bidegree $(p,q)$.  Put $P=\{n\in\Z\st f(n)=0\}$
and $Q=\{n\in\Z\st g(n)=0\}$ (so $|P|\leq p$ and $|Q|\leq q$, and
often $P$ and $Q$ will be empty).  Suppose for the moment that $p<q$.
If $P=\emptyset$ we will show that $K=K_0=0$.  If $P\neq\emptyset$
then the most common situation is that $\dim(K)=1$ and $\dim(K_0)=0$,
but it will take a little work to formulate a precise statement.  We
put 
\[ R = \{i\in\Z\st \exists j\in P \text{ with }
          j > i \text{ and } \{i,i+1,\dotsc,j-1\}\cap Q=\emptyset
   \}.
\]

\begin{proposition}\label{prop-ann-S}\leavevmode
 \begin{itemize}
  \item[(a)] The restriction map $K\to\Map(R,\C)$ is zero.
  \item[(b)] The restriction map $K\to\Map(P\sm R,\C)$ is an
   isomorphism.
  \item[(c)] We have $\dim(K)=|P\sm R|\leq\min(|P|,|Q|+1)$.
  \item[(d)] If $\max(P)\leq\max(Q)$ then $K=K_0$.  Otherwise there is a
   unique element $b\in K$ with $b_{\max(P)}=1$ and $b_i=0$ for
   $i<\max(P)$, and we have $K=K_0\oplus\C b$.
 \end{itemize}
\end{proposition}
\begin{proof}
 First note that $K$ is just the space of rapidly decreasing sequences
 $a$ satisfying $f(k-1)a_{k-1}=g(k)a_k$ for all $k$.   

 Suppose that $a\in K$ and $i\in R$, so there exists $j>i$ with
 $g(j)=0$ and $f(k)\neq 0$ for $i\leq k<j$.  The recurrence relation
 gives 
 \[ f(i)f(i+1)\dotsb f(j-1)a_i = g(i+1)g(i+2)\dotsb g(j)a_j, \]
 from which we deduce that $a_i=0$.  This proves~(a).

 Next, note that for $k\ll 0$ we will have $f(k-1),g(k)\neq 0$ so we
 can write the recurrence relation as $a_{k-1}=a_k\,g(k)/f(k-1)$.  As
 $p<q$ we have $|g(k)/f(k-1)|\to\infty$ as $k\to -\infty$.  Thus, the
 only way the sequence can be rapidly decreasing is if $a_k=0$ for
 $k\ll 0$.  Now suppose that $a_{k-1}=0$; we claim that $a_k$ is also
 zero.  If $k\in R$ then this holds by part~(a), if $k\in P\sm R$ then
 it holds by assumption, and if $k\not\in P$ then it follows from the
 relation $f(k-1)a_{k-1}=g(k)a_k$.  It now follows by induction that
 $a=0$, so the restriction $K\to\Map(P\sm R,\C)$ is injective.  

 Now suppose we have $i\in P\sm R$.  If $i$ is maximal in $P$, we put 
 \[ b_{ik} = \begin{cases}
     0 & \text{ if } k < i \\
     \prod_{j=i+1}^k \frac{f(j-1)}{g(j)} & \text{ if } k \geq i.
    \end{cases}
 \]
 This gives an element $b_i\in K$, which lies in $K_0$ iff
 $\max(Q)\geq i=\max(P)$.  We are using the standard convention that
 the empty product is one, so $b_{ii}=1$, but $b_{ij}=0$ for all
 $j\in P\sm\{i\}$.

 Suppose instead that $i$ is not maximal in
 $P$, and let $j$ be the smallest element in $P$ with $j>i$.  As
 $i\not\in R$ the set $\{i,i+1,\dotsc,j-1\}\cap Q$ must be nonempty;
 let $m$ be the smallest element.  Put 
 \[ b_{ik} = \begin{cases}
     0 & \text{ if } k < i \text{ or } k > m \\
     \prod_{j=i+1}^k \frac{f(j-1)}{g(j)} & \text{ if } i\leq k\leq m.
    \end{cases}
 \]
 Again we have $b_i\in K$ with $b_{ii}=1$ and $b_{ij}=0$ for
 $j\in P\sm\{i\}$.

 All claims are now clear except for the fact that
 $|P\sm R|\leq |Q|+1$.  This holds because every element of $P\sm R$
 is either maximal in $P$ or dominated by an element of $Q$.  
\end{proof}

\begin{remark}
 Suppose that $L=z\,f(\Dl)-g(\Dl)$ and $L^*=z\,g(-\Dl)-f(-\Dl)$.  We
 find that the map $(a_n)_{n\in\Z}\to (a_{-n})_{n\in\Z}$ gives an
 isomorphism $\ann(L,S)\simeq\ann(L^*,S)$.  Using this we can
 understand $\ann(L,S)$ in the case where $p>q$.  The case where $p=q$
 will require a slightly different approach.
\end{remark}

\begin{definition}
 Suppose that $L=z\,f(\Dl)-g(\Dl)$, where 
 \begin{align*}
  f(t) &= \al\prod_{j=1}^p(t-a_j) \\
  g(t) &= \bt\prod_{j=1}^q(t-b_j).
 \end{align*}
 Suppose that $m\in\C$ is such that $\exp(m)=(-1)^p\al/\bt$.  We then
 put 
 \[ v(s) = v_{L,m}(s) =
    e^{ms} \prod_{j=1}^p\Gm(a_i-s+1)^{-1} 
           \prod_{j=1}^q\Gm(s+1-b_i)^{-1}.
 \]
 Recall that the Gamma function has poles but no zeros, so $v(s)$ is
 holomorphic.  
\end{definition}

\begin{proposition}
 The map $w(z)\mapsto w(e^{2\pi i s})v(s)$ gives an isomorphism from
 the space of meromorphic functions on $\C^\tm$ to $\ann(L,F)$.
\end{proposition}
\begin{proof}
 It is not hard to see that a nonzero meromorphic function $u\in F$
 satisfies $Lu=0$ if and only if $u(s)/u(s-1)=f(s-1)/g(s)$.  Using the
 functional equation $x\,\Gm(x)=\Gm(x+1)$ one can check that $v(s)$
 has the above property, so $u\in\ann(L,F)$.  If $v$ is another
 element of $\ann(L,F)$ then we have $(u/v)(s)=(u/v)(s-1)$, so
 $(u/v)(s)=w(e^{2\pi is})$ for some holomorphic function on $\C^\tm$,
 as claimed.
\end{proof}

To understand the nature of $v_{L,m}(s)$ and related functions, we
need to know about the asymptotics of the Gamma function.  

\begin{bibdiv}
\begin{biblist}
\bibselect{%
../../BiBTeX/refs,%
../../BiBTeX/myrefs%
}
\end{biblist}
\end{bibdiv}

\end{document}
