\documentclass[reqno]{amsart}
%\usepackage{hyperref}
\usepackage{fullpage}
\usepackage{amsrefs}
\usepackage{verbatim}
\usepackage{tikz}
\usepackage{graphicx}

\newif\ifscreen
\newif\iftwo
\newif\ifshowall
\newif\ifshowkeys
\screenfalse
\twotrue
\showallfalse
\showkeystrue

\input xypic

\ifshowkeys
\newcommand{\lbl}[1]{\label{#1}\textup{[\texttt{#1}]}\ \\}
\else
\newcommand{\lbl}{\label}
\fi

\definecolor{olivegreen}{cmyk}{0.64,0,0.95,0.40}
\definecolor{rawsienna}{cmyk}{0,0.72,1,0.45}
\definecolor{lightgreen}{rgb}{0.85,1.0,0.85}

\newcommand{\GREENYELLOW}[1]{{\color{greenyellow}#1}}
\newcommand{\YELLOW}[1]{{\color{yellow}#1}}
\newcommand{\YLW}[1]{{\color{yellow}#1}}
\newcommand{\GOLDENROD}[1]{{\color{goldenrod}#1}}
\newcommand{\DANDELION}[1]{{\color{dandelion}#1}}
\newcommand{\APRICOT}[1]{{\color{apricot}#1}}
\newcommand{\PEACH}[1]{{\color{peach}#1}}
\newcommand{\MELON}[1]{{\color{melon}#1}}
\newcommand{\YELLOWORANGE}[1]{{\color{yelloworange}#1}}
\newcommand{\ORANGE}[1]{{\color{orange}#1}}
\newcommand{\BURNTORANGE}[1]{{\color{burntorange}#1}}
\newcommand{\BITTERSWEET}[1]{{\color{bittersweet}#1}}
\newcommand{\REDORANGE}[1]{{\color{redorange}#1}}
\newcommand{\MAHOGANY}[1]{{\color{mahogany}#1}}
\newcommand{\MAROON}[1]{{\color{maroon}#1}}
\newcommand{\BRICKRED}[1]{{\color{brickred}#1}}
\newcommand{\RED}[1]{{\color{red}#1}}
\newcommand{\ORANGERED}[1]{{\color{orangered}#1}}
\newcommand{\RUBINERED}[1]{{\color{rubinered}#1}}
\newcommand{\WILDSTRAWBERRY}[1]{{\color{wildstrawberry}#1}}
\newcommand{\SALMON}[1]{{\color{salmon}#1}}
\newcommand{\CARNATIONPINK}[1]{{\color{carnationpink}#1}}
\newcommand{\MAGENTA}[1]{{\color{magenta}#1}}
\newcommand{\VIOLETRED}[1]{{\color{violetred}#1}}
\newcommand{\RHODAMINE}[1]{{\color{rhodamine}#1}}
\newcommand{\MULBERRY}[1]{{\color{mulberry}#1}}
\newcommand{\REDVIOLET}[1]{{\color{redviolet}#1}}
\newcommand{\FUCHSIA}[1]{{\color{fuchsia}#1}}
\newcommand{\LAVENDER}[1]{{\color{lavender}#1}}
\newcommand{\THISTLE}[1]{{\color{thistle}#1}}
\newcommand{\ORCHID}[1]{{\color{orchid}#1}}
\newcommand{\DARKORCHID}[1]{{\color{darkorchid}#1}}
\newcommand{\PURPLE}[1]{{\color{purple}#1}}
\newcommand{\PLUM}[1]{{\color{plum}#1}}
\newcommand{\VIOLET}[1]{{\color{violet}#1}}
\newcommand{\ROYALPURPLE}[1]{{\color{royalpurple}#1}}
\newcommand{\BLUEVIOLET}[1]{{\color{blueviolet}#1}}
\newcommand{\PERIWINKLE}[1]{{\color{periwinkle}#1}}
\newcommand{\CADETBLUE}[1]{{\color{cadetblue}#1}}
\newcommand{\CORNFLOWERBLUE}[1]{{\color{cornflowerblue}#1}}
\newcommand{\MIDNIGHTBLUE}[1]{{\color{midnightblue}#1}}
\newcommand{\NAVYBLUE}[1]{{\color{navyblue}#1}}
\newcommand{\ROYALBLUE}[1]{{\color{royalblue}#1}}
\newcommand{\BLU}[1]{{\color{blue}#1}}
\newcommand{\BLUE}[1]{{\color{blue}#1}}
\newcommand{\CERULEAN}[1]{{\color{cerulean}#1}}
\newcommand{\CYAN}[1]{{\color{cyan}#1}}
\newcommand{\PROCESSBLUE}[1]{{\color{processblue}#1}}
\newcommand{\SKYBLUE}[1]{{\color{skyblue}#1}}
\newcommand{\TURQUOISE}[1]{{\color{turquoise}#1}}
\newcommand{\TEALBLUE}[1]{{\color{tealblue}#1}}
\newcommand{\AQUAMARINE}[1]{{\color{aquamarine}#1}}
\newcommand{\BLUEGREEN}[1]{{\color{bluegreen}#1}}
\newcommand{\EMERALD}[1]{{\color{emerald}#1}}
\newcommand{\JUNGLEGREEN}[1]{{\color{junglegreen}#1}}
\newcommand{\SEAGREEN}[1]{{\color{seagreen}#1}}
\newcommand{\GREEN}[1]{{\color{green}#1}}
\newcommand{\FORESTGREEN}[1]{{\color{forestgreen}#1}}
\newcommand{\PINEGREEN}[1]{{\color{pinegreen}#1}}
\newcommand{\LIMEGREEN}[1]{{\color{limegreen}#1}}
\newcommand{\YELLOWGREEN}[1]{{\color{yellowgreen}#1}}
\newcommand{\SPRINGGREEN}[1]{{\color{springgreen}#1}}
\newcommand{\OLIVEGREEN}[1]{{\color{olivegreen}#1}}
\newcommand{\OLG}[1]{{\color{olivegreen}#1}}
\newcommand{\RAWSIENNA}[1]{{\color{rawsienna}#1}}
\newcommand{\SEPIA}[1]{{\color{sepia}#1}}
\newcommand{\BROWN}[1]{{\color{brown}#1}}
\newcommand{\TAN}[1]{{\color{tan}#1}}
\newcommand{\GRAY}[1]{{\color{gray}#1}}
\newcommand{\LGRAY}[1]{{\color{gray!40}#1}}
\newcommand{\WHITE}[1]{{\color{white}#1}}
\newcommand{\BLACK}[1]{{\color{black}#1}}

\newcommand{\bbm}       {\left[\begin{matrix}}
\newcommand{\ebm}       {\end{matrix}\right]}
\newcommand{\bsm}       {\left[\begin{smallmatrix}}
\newcommand{\esm}       {\end{smallmatrix}\right]}
\newcommand{\bpm}       {\begin{pmatrix}}
\newcommand{\epm}       {\end{pmatrix}}
\newcommand{\bcf}[2]{\left(\begin{array}{c}{#1}\\{#2}\end{array}\right)}

\newcommand{\adj}       {\operatorname{adj}}
\newcommand{\ann}       {\operatorname{ann}}
\newcommand{\diag}      {\operatorname{diag}}
\newcommand{\img}       {\operatorname{img}}
\newcommand{\rnk}       {\operatorname{rank}}
\newcommand{\sgn}       {\operatorname{sgn}}
\newcommand{\spn}       {\operatorname{span}}
\newcommand{\trc}       {\operatorname{trace}}

\newcommand{\pp}{\hphantom{+}}
\newcommand{\tm}{\times}
\newcommand{\sse}{\subseteq}
\newcommand{\st}{\;|\;}
\newcommand{\sm}{\setminus}
\newcommand{\iffa}      {\Leftrightarrow}
\newcommand{\xra}{\xrightarrow}
\newcommand{\xla}{\xleftarrow}

\newcommand{\half}{\tfrac{1}{2}}

\newcommand{\N}         {{\mathbb{N}}}
\newcommand{\Z}         {{\mathbb{Z}}}
\newcommand{\Q}         {{\mathbb{Q}}}
\newcommand{\R}         {{\mathbb{R}}}
\newcommand{\C}         {{\mathbb{C}}}

\newcommand{\va}        {\mathbf{a}}
\newcommand{\vb}        {\mathbf{b}}
\newcommand{\vc}        {\mathbf{c}}
\newcommand{\vd}        {\mathbf{d}}
\newcommand{\ve}        {\mathbf{e}}
\newcommand{\vf}        {\mathbf{f}}
\newcommand{\vg}        {\mathbf{g}}
\newcommand{\vh}        {\mathbf{h}}
\newcommand{\vi}        {\mathbf{i}}
\newcommand{\vj}        {\mathbf{j}}
\newcommand{\vk}        {\mathbf{k}}
\newcommand{\vl}        {\mathbf{l}}
\newcommand{\vm}        {\mathbf{m}}
\newcommand{\vn}        {\mathbf{n}}
\newcommand{\vo}        {\mathbf{o}}
\newcommand{\vp}        {\mathbf{p}}
\newcommand{\vq}        {\mathbf{q}}
\newcommand{\vr}        {\mathbf{r}}
\newcommand{\vs}        {\mathbf{s}}
\newcommand{\vt}        {\mathbf{t}}
\newcommand{\vu}        {\mathbf{u}}
\newcommand{\vv}        {\mathbf{v}}
\newcommand{\vw}        {\mathbf{w}}
\newcommand{\vx}        {\mathbf{x}}
\newcommand{\vy}        {\mathbf{y}}
\newcommand{\vz}        {\mathbf{z}}

\newcommand{\vA}        {\mathbf{A}}
\newcommand{\vB}        {\mathbf{B}}
\newcommand{\vC}        {\mathbf{C}}
\newcommand{\vD}        {\mathbf{D}}
\newcommand{\vE}        {\mathbf{E}}
\newcommand{\vF}        {\mathbf{F}}
\newcommand{\vG}        {\mathbf{G}}
\newcommand{\vH}        {\mathbf{H}}
\newcommand{\vI}        {\mathbf{I}}
\newcommand{\vJ}        {\mathbf{J}}
\newcommand{\vK}        {\mathbf{K}}
\newcommand{\vL}        {\mathbf{L}}
\newcommand{\vM}        {\mathbf{M}}
\newcommand{\vN}        {\mathbf{N}}
\newcommand{\vO}        {\mathbf{O}}
\newcommand{\vP}        {\mathbf{P}}
\newcommand{\vQ}        {\mathbf{Q}}
\newcommand{\vR}        {\mathbf{R}}
\newcommand{\vS}        {\mathbf{S}}
\newcommand{\vT}        {\mathbf{T}}
\newcommand{\vU}        {\mathbf{U}}
\newcommand{\vV}        {\mathbf{V}}
\newcommand{\vW}        {\mathbf{W}}
\newcommand{\vX}        {\mathbf{X}}
\newcommand{\vY}        {\mathbf{Y}}
\newcommand{\vZ}        {\mathbf{Z}}

\newcommand{\al}        {\alpha}
\newcommand{\bt}        {\beta} 
\newcommand{\gm}        {\gamma}
\newcommand{\dl}        {\delta}
\newcommand{\ep}        {\epsilon}
\newcommand{\zt}        {\zeta}
\newcommand{\et}        {\eta}
\newcommand{\tht}       {\theta}
\newcommand{\io}        {\iota}
\newcommand{\kp}        {\kappa}
\newcommand{\lm}        {\lambda}
\newcommand{\ph}        {\phi}
\newcommand{\ch}        {\chi}
\newcommand{\ps}        {\psi}
\newcommand{\rh}        {\rho}
\newcommand{\sg}        {\sigma}
\newcommand{\om}        {\omega}

\newcommand{\Gm}        {\Gamma}
\newcommand{\Dl}        {\Delta}

\newcommand{\CA}        {\mathcal{A}}
\newcommand{\CB}        {\mathcal{B}}
\newcommand{\CC}        {\mathcal{C}}
\newcommand{\CD}        {\mathcal{D}}
\newcommand{\CE}        {\mathcal{E}}
\newcommand{\CF}        {\mathcal{F}}
\newcommand{\CG}        {\mathcal{G}}
\newcommand{\CH}        {\mathcal{H}}
\newcommand{\CI}        {\mathcal{I}}
\newcommand{\CJ}        {\mathcal{J}}
\newcommand{\CK}        {\mathcal{K}}
\newcommand{\CL}        {\mathcal{L}}
\newcommand{\CM}        {\mathcal{M}}
\newcommand{\CN}        {\mathcal{N}}
\newcommand{\CO}        {\mathcal{O}}
\newcommand{\CP}        {\mathcal{P}}
\newcommand{\CQ}        {\mathcal{Q}}
\newcommand{\CR}        {\mathcal{R}}
\newcommand{\CS}        {\mathcal{S}}
\newcommand{\CT}        {\mathcal{T}}
\newcommand{\CU}        {\mathcal{U}}
\newcommand{\CV}        {\mathcal{V}}
\newcommand{\CW}        {\mathcal{W}}
\newcommand{\CX}        {\mathcal{X}}
\newcommand{\CY}        {\mathcal{Y}}
\newcommand{\CZ}        {\mathcal{Z}}


\newcommand{\ov}        {\overline}
\newcommand{\ip}[1]     {\langle #1\rangle}
\renewcommand{\ss}      {\scriptstyle}

\renewcommand{\:}       {\colon}

\newcommand{\barmat}[2]{\left[\begin{array}{c|c}\!\!\raisebox{0pt}[0.45cm][0.35cm]{$#1$} & \raisebox{0pt}[0.45cm][0.35cm]{$#2$}\!\!\end{array}\right]}

\newcommand{\eqpair}[4]{\begin{array}{rl} #1 &= #2 \\ #3 &= #4\end{array}}

\newcommand{\han}[1]{\begin{CJK*}{UTF8}{zhsong}\BLUE{#1}\end{CJK*}}
\newcommand{\bhan}[1]{(\begin{CJK*}{UTF8}{zhsong}\BLUE{#1}\end{CJK*})}

\newcommand{\EMPH}[1]{\emph{\RED{#1}}}
\newcommand{\DEFN}[1]{\emph{\PURPLE{#1}}}
\newcommand{\VEC}[1]    {\mathbf{#1}}

\newcommand{\ghost}{{\tiny $\color[rgb]{1,1,1}.$}}

\newcommand{\reminderbar}{\par\medskip\par\hrule\par\medskip\par}

\newcommand{\uc}{\uncover}

\newcommand{\bbox}[1]{
\[ \mbox{\begin{tikzpicture}%
   \draw(0,0) node[draw,thick,olivegreen,rectangle] {\color{black} #1};%
  \end{tikzpicture}} \]
}

\newcommand{\cbox}[1]{
\begin{center}\begin{tikzpicture}%
   \draw(0,0) node[draw,thick,olivegreen,rectangle] {\color{black} #1};%
\end{tikzpicture}\end{center}
}



\newtheorem{theorem}{Theorem}[section]
\newtheorem{conj}[theorem]{Conjecture}
\newtheorem{lemma}[theorem]{Lemma}
\newtheorem{proposition}[theorem]{Proposition}
\newtheorem{corollary}[theorem]{Corollary}
\theoremstyle{definition}
\newtheorem{remark}[theorem]{Remark}
\newtheorem{predefinition}[theorem]{Predefinition}
\newtheorem{definition}[theorem]{Definition}
\newtheorem{example}[theorem]{Example}
\newtheorem{algorithm}[theorem]{Algorithm}
\newtheorem{method}[theorem]{Method}
\newtheorem{fact}[theorem]{Fact}
% Exercises are numbered separately
\newtheorem{exercise}{Exercise}[section]

%\renewenvironment{solution}{\SolutionAtEnd}{\endSolutionAtEnd}
%\renewenvironment{solution}{\SolutionHidden}{\endSolutionHidden}

\newwrite\refs
\openout\refs=\jobname.refs
\makeatletter
\renewcommand\@setref[3]{%
        \ifx#1\relax
                \write\refs{'#3' \thepage\space undefined}%
                \protect \G@refundefinedtrue
                \nfss@text{\reset@font\bfseries ??}%
                \@latex@warning{Reference `#3' on page \thepage\space
                                undefined}%
        \else
                \begingroup
                \count@\expandafter\@secondoftwo#1\relax
                \ifnum\c@page<\count@
                        \write\refs{'#3' \thepage\space
                                    \expandafter\@secondoftwo#1}%
                \fi
                \endgroup
                \expandafter#2#1\null
        \fi
}
\makeatother

\newcommand{\dfn}[1]{\emph{{#1}}\index{#1}}
\newcommand{\idx}[1]{{#1}\index{#1}}

\makeindex

\begin{document}

\title{MAS290 Methods for Differential Equations}
\author{Neil Strickland}

\maketitle

\tableofcontents 

\section{Introduction}
\label{sec-intro}

\section{First order autonomous ODEs}

In this section we discuss systems of differential equations like these.

\begin{example}
 Suppose that the number of fish in the sea is $F$, and the number of
 sharks is $S$.  The sharks breed, the fish breed, and the sharks eat
 the fish.  The rates of change of $F$ and $S$ are given by 
 \[ \frac{dF}{dt} = (\al - \bt S)F \hspace{5em}
    \frac{dS}{dt} = -(\gm - \dl F)S, 
 \]
 for some constants $\al$, $\bt$, $\gm$ and $\dl$.  These are called
 the \emph{Lotka Volterra equations}.  We will discuss later why they
 are a good model.
\end{example}

\begin{example}
 Some aspects of the weather are described by the \emph{Lorenz equations}:
 \[ \frac{dx}{dt} = 10(y-x) \hspace{5em}
    \frac{dy}{dt} = (28-z)x-y \hspace{5em}
    \frac{dz}{dt} = xy - \tfrac{8}{3}z.
 \]
\end{example}

\begin{example}
 Perhaps explain a simplified version of the Hodgkin-Huxley model.
\end{example}

\begin{example}
 The one-sector growth model is a complex system of differential
 equations that relates capital investment, labour costs, interest
 rates, savings rates, profits and similar variables.
\end{example}

\begin{itemize}
 \item All of these are \emph{ordinary} differential equations; there
  are derivatives with respect to time, but not with respect to other
  variables.  This is different from the equations that govern the
  flow of heat along an iron bar, for example.  Heat will only flow if
  some parts of the bar are hotter than others, so the equations
  involve the derivatives of temperature with respect to position as
  well as with respect to time.  Equations like this are called
  \emph{partial differential equations}; you will study them in later
  courses.
 \item All the above equations are of \emph{first order}; on the left
  hand side we have only the first derivative $dx/dt$, not higher
  derivatives like $d^2x/dt^2$.  This is different from most of the
  equations of physics, which usually involve acceleration, which is a
  second derivative.  However, there is a way to convert second order
  equations to first order systems, which we will discuss later.
 \item All the above equations are \emph{deterministic}; there are no
  random effects.  The theory of \emph{stochastic differential
   equations} includes random effects, which are important in
  economics and finance.  You will be able to study these in your
  final year in Sheffield.
 \item All the above equations are \emph{autonomous}; they do not
  include external effects.  For example, a more realistic description
  of populations of sharks and fish would include the external effect
  of humans catching fish.
\end{itemize}

\section{Review of differential equations in one variable}

Suppose that we have a variable $x$ that depends on a variable $t$,
and we write $\dot{x}$ for $dx/dt$.  If we know that $x(t)$ satisfies
a differential equation like $\dot{x}(t)=f(x(t))$, then we may be able
to deduce a formula for $x(t)$.

\begin{example}
 Suppose we know that $\dot{x}=\lm x$ for some constant $\lm$, and
 that $x=x_0$ when $t=0$.  It is then well-known that
 $x=e^{\lm t}x_0$ for all $t$.  To check this, put $y=e^{-\lm t}x$.
 Using the product rule and the equation $\dot{x}=\lm x$ we get
 \[ \dot{y}=\frac{d}{dt}(e^{-\lm t}x) = 
     x \frac{d}{dt} e^{-\lm t} + e^{-\lm t} \frac{dx}{dt} = 
     -\lm x e^{-\lm t} + e^{-\lm t}\lm x = 0.
 \]
 As $\dot{y}=0$, we see that $y$ is constant.  At $t=0$ we have
 $y=e^0x_0=x_0$, so we must have $y=x_0$ for all time.  As
 $e^{-\lm t}x=y=x_0$, it follows that $x=e^{\lm t}x_0$ as claimed.

 We will need to recall the shape of the graph of $e^{\lm t}$.  If
 $\lm>0$, then it looks like the picture on the left below. It is
 always positive, and never crosses the $x$-axis.  As
 $t\to +\infty$, the function $e^{\lm t}$ grows large very quickly.
 \begin{center}
  \begin{tikzpicture}
   \begin{scope}
    \draw[->] (-2.1,0) -- (2.1,0);
    \draw[->] (0,-0.1) -- (0,4.1);
    \draw[red,domain=-2:2,samples=300,smooth,variable=\x]
     plot({\x},{0.5*exp(\x)});
    \draw (-1,3.8) node{$\lm > 0$};
   \end{scope}
   \begin{scope}[xshift=8cm]
    \draw[->] (-2.1,0) -- (2.1,0);
    \draw[->] (0,-0.1) -- (0,4.1);
    \draw[red,domain=-2:2,samples=300,smooth,variable=\x]
     plot({\x},{0.5*exp(-\x)});
    \draw (1,3.8) node{$\lm < 0$};
   \end{scope}
  \end{tikzpicture}
 \end{center}
 If $\lm<0$ then the graph of $e^{\lm t}$ looks like the picture on
 the right above.  Again, it is always positive, and never crosses the
 $x$-axis.  It approaches zero as $t\to +\infty$.
\end{example}

\begin{example}
 If $\dot{x}=\sqrt{x}$ then it works out that $x=(t+c)^2/4$ for some
 constant $c$.  To see this, put $y=2\sqrt{x}=2x^{1/2}$, so $x=y^2/4$.
 We then have 
 \[ \dot{y} = 2 \tm \half \tm x^{\half - 1} \dot{x}
     = x^{-1/2} \dot{x} = x^{-1/2} x^{1/2} = 1.
 \]
 This means that $y=t+c$ for some constant $c$, so $x=(t+c)^2/4$.
\end{example}

\begin{remark}
 The above two examples ($\dot{x}=\lm x$ and $\dot{x}=\sqrt{x}$) are
 first order differential equations in one variable.  \emph{First
  order} means that only the first derivative $\dot{x}=dx/dt$ is
 involved, not the second derivative $\ddot{x}=d^2x/dt^2$ or any
 higher derivatives.  An example of a second order equation is
 $\ddot{x}+\om^2x=0$, with solution $x=A\cos(\om t)+B\sin(\om t)$.  In
 the first half of this course we will mostly study first order
 equations.  
\end{remark}

\begin{remark}
 The equations $\dot{x}=\lm x$ and $\dot{x}=\sqrt{x}$ are also
 \emph{autonomous}, which means that the variable $t$ does not appear
 explicitly.  An example of a non-autonomous equation is
 $\dot{x}=x+t$, with solution $x=Ae^t-1-t$.  In this course we will
 mainly study autonomous equations.
\end{remark}

\begin{definition}
 Consider a second order, autonomous differential equation
 $\ddot{x}+b\dot{x}+cx=0$.  The \emph{auxiliary polynomial} is
 $p(\lm)=\lm^2+b\lm+c=0$.  The roots of the equation $p(\lm)=0$ are
 $(-b\pm\sqrt{b^2-4c})/2$. 
 \begin{itemize}
  \item[(a)] If $b^2\geq 4c$ then the roots
   $\lm_1=(-b-\sqrt{b^2-4c})/2$ and $\lm_2=(-b+\sqrt{b^2-4c})/2$ are
   both real numbers.
  \item[(b)] If $b^2<4c$ then the numbers $\lm=-b/2$ and
   $\om=\sqrt{4c-b^2}/2$ are real, and the two roots of the auxiliary
   equation are $\lm-i\om$ and $\lm+i\om$.  We call $\lm$ the
   \emph{growth rate}, and $\om$ the \emph{angular frequency}.
 \end{itemize}
\end{definition}

\begin{proposition}\leavevmode
 \begin{itemize}
  \item[(a)] If $b^2\geq 4c$ then the solutions of the equation
   $\ddot{x}+b\dot{x}+cx=0$ are of the form
   $x=A_1e^{\lm_1t}+A_2e^{\lm_2t}$ (where $A_1$ and $A_2$ are real
   constants). 
  \item[(b)] If $b^2<4c$ then the solutions are of the form
   $(B\cos(\om t)+C\sin(\om t))e^{\lm t}$ (where $B$ and $C$ are real
   constants). 
 \end{itemize}
\end{proposition}
\begin{proof}
 The first thing is to prove that the above formulae do in fact
 satisfy the equation $\ddot{x}+b\dot{x}+cx=0$.  In case~(a) we have
 \begin{align*}
  x        &= A_1e^{\lm_1t} + A_2e^{\lm_2t} \\
  \dot{x}  &= A_1\lm_1e^{\lm_1t} + A_2\lm_2e^{\lm_2t} \\
  \ddot{x} &= A_1\lm_1^2e^{\lm_1t} + A_2\lm_2^2e^{\lm_2t} \\
  \ddot{x}+b\dot{x}+cx &= 
   A_1(\lm_1^2+b\lm_1+c)e^{\lm_1t} + A_2(\lm_2^2+b\lm_2+c)e^{\lm_2t} \\
    &= A_1p(\lm_1)e^{\lm_1t} + A_2p(\lm_2)e^{\lm_2t} = 0,
 \end{align*}
 as required.  In case~(b), we recall that 
 \begin{align*}
  \cos(\om t) &= (e^{i\om t}+e^{-i\om t})/2 \\
  \sin(\om t) &= -i(e^{i\om t}-e^{-i\om t})/2,
 \end{align*}
 so 
 \begin{align*}
  x &= (B\cos(\om t)+C\sin(\om t))e^{\lm t} \\
    &= (Be^{i\om t}+Be^{-i\om t}-iCe^{i\om t}+iCe^{-i\om t})e^{\lm t}/2 \\
    &= \frac{B-iC}{2}e^{(\lm+i\om)t} + \frac{B+iC}{2}e^{(\lm-i\om)t}.
 \end{align*}
 In other words, if we put $A_1=(B-iC)/2$ and $A_2=(B+iC)/2$ and
 $\lm_1=\lm+i\om$ and $\lm_2=\lm-i\om$, then $\lm_1$ and $\lm_2$ are
 the roots of the auxiliary polynomial, and
 $x=A_1e^{\lm_1t}+A_2e^{\lm_2t}$.  Thus, we have
 $\ddot{x}+b\dot{x}+cx=0$ just as in case~(a); the only difference is
 that some of the numbers involved are complex, but that does not
 matter. 

 The more difficult problem is to prove that \emph{every} solution has
 the form described above.  Suppose that $x(t)$ is a function of $t$
 such that $\ddot{x}+b\dot{x}+cx=0$.  Let $\lm_1$ and $\lm_2$ be the
 roots of the auxiliary polynomial, so 
 \[ t^2+bt+c = (t-\lm_1)(t-\lm_2) = t^2-(\lm_1+\lm_2)t+\lm_1\lm_2, \]
 so $\lm_1+\lm_2=-b$ and $\lm_1\lm_2=c$.  Put
 \begin{align*}
  P_1(t) &= e^{-\lm_1t}(\dot{x}(t) - \lm_2 x(t)) \\
  P_2(t) &= e^{-\lm_2t}(\dot{x}(t) - \lm_1 x(t)).
 \end{align*}
 We then have
 \begin{align*}
  \dot{P}_1 &= 
   -\lm_1 e^{-\lm_1t}(\dot{x} - \lm_2 x) + 
    e^{-\lm_1t}(\ddot{x} - \lm_2\dot{x}) \\
   &= e^{-\lm_1t}(\ddot{x} - (\lm_1+\lm_2)\dot{x} +\lm_1\lm_2 x) 
    = e^{-\lm_1t}(\ddot{x} + b\dot{x} + cx)
    = 0. 
 \end{align*}
 Thus, $P_1$ is actually constant.  By a similar calculation, $P_2$ is
 also constant.  We can also rearrange the definition of $P_1$ and
 $P_2$ to get
 \begin{align*}
  \dot{x} - \lm_2 x &= P_1 e^{\lm_1t} \\
  \dot{x} - \lm_1 x &= P_2 e^{\lm_2t}.
 \end{align*}
 Subtracting these equations and rearranging gives 
 \[ x = \frac{P_1}{\lm_1-\lm_2} e^{\lm_1t} + 
        \frac{P_2}{\lm_2-\lm_1} e^{\lm_2t}.
 \]
 In other words, if we put $A_1=P_1/(\lm_1-\lm_2)$ and
 $A_2=P_2/(\lm_2-\lm_1)$ then $x=A_1e^{\lm_1t}+A_2e^{\lm_2t}$ as
 required.  All this works perfectly well even in case~(b) where the
 numbers $\lm_k$, $P_k$ and $A_k$ may be complex.
\end{proof}

\begin{remark}
 In many applications, it works out that $\lm<0$ and $\om$ is much
 larger than $|\lm|$.  In that case, the graph of the function
 $e^{\lm t}\sin(\om t)$ is like this:
 \begin{center}
  \begin{tikzpicture}
   \begin{scope}
    \draw[->] (-0.1,0) -- (6.1,0);
    \draw[->] (0,-1.1) -- (0,1.1);
    \draw[red,domain=-0.1:6,samples=600,smooth,variable=\t]
     plot({\t},{sin(1800*\t)*exp(-0.3*\t)});
   \end{scope}
  \end{tikzpicture}
 \end{center}

\end{remark}

\section{Phase portraits}

Consider a differential system 
\[ \dot{x}(t) = f(x(t),y(t)) \hspace{ 5em }
   \dot{y}(t) = g(x(t),y(t)).
\]
If we are given initial conditions $x(0)=x_0$ and $y(0)=y_0$, we can 
solve the equations to find $x(t)$ and $y(t)$ for all $t$ (or at
least, for some range of values of $t$).  The points $(x(t),y(t))$
will then trace out a curve in the plane, passing through the initial
point $(x_0,y_0)$.  If we change the initial point then we will
usually get a different curve.  The collection of all solution curves
(which cover the whole plane) is called the \emph{phase portrait} of
the system.  

\begin{example}
 Phase portraits for three very simple systems are shown below.
 \begin{center}
  \begin{tikzpicture}[scale=1.4]
   \begin{scope}
    \draw (0,1.3) node {$\dot{x}=1,\;\dot{y}=0$};
    \foreach \y in {-1,-0.9,...,1} {
     \draw[red] (-1,\y) -- (1,\y); 
    }
   \end{scope}
   \begin{scope}[xshift=4cm]
    \draw (0,1.3) node {$\dot{x}=0,\;\dot{y}=1$};
    \foreach \x in {-1,-0.9,...,1} {
     \draw[red] (\x,-1) -- (\x,1); 
    }
   \end{scope}
   \begin{scope}[xshift=8cm]
    \draw (0,1.3) node {$\dot{x}=-y,\;\dot{y}=x$};
    \foreach \r in {0.1,0.2,...,1} {
     \draw[red] (0,0) circle(\r);
    }
   \end{scope}
  \end{tikzpicture}
 \end{center}
 For the system $\dot{x}=1,\dot{y}=0$ the solution is just $x=x_0+t$
 and $y=y_0$, which gives a horizontal line.  Similarly, for the
 system $\dot{x}=0,\dot{y}=1$ the phase portrait just consists of
 vertical lines.  For the third system, we can choose any $r\geq 0$
 and then the functions $x=r\cos(t)$ and $y=r\sin(t)$ give a
 solution.  The corresponding curve is a circle of radius $r$ centred
 at the origin.  All these circles together form the phase portrait.
\end{example}

Often we can sketch the phase portrait and deduce useful information
about the behaviour of the differential system, even if we cannot find
a formula for the solutions.

\section{Critical points}

\section*{To do}

In this section we consider differential equations of the form
$\dot{x}=u(x)$, where $x\:\R\to\R^n$ and $u\:\R^n\to\R^n$.  The main
topic is the behaviour near equilibrium points $a\in\R^n$ where
$u(a)=0$.  We can reduce to the case where $a=0$, so 
$u(x)\simeq Ax+O(\|x\|^2)$ for some matrix $A$.  If $x(0)=c\simeq 0$
then for small $t$ we have $\dot{x}\simeq Ax$ so $x\simeq e^{At}c$.
This means that $0$ is a stable equilibrium if all the eigenvalues of
$A$ lie in the left half plane.

For more global stability analysis we seek Lyapunov functions $V(x)$
such that $V^{-1}\{0\}=0$ and $u.\nabla V<0$ away from the origin, so
that $(V\circ x)'\leq 0$ for all solutions to our ODE.

We also need to cover other material about sketching trajectories.

\section{Second order linear ODEs}

Here we need to cover power series methods, classification of singular
points, Sturm-Liouville theory.

\end{document}
